\documentclass{article}
% \usepackage[margin=1.5cm,landscape]{geometry}
\usepackage{proof}
\usepackage{amssymb}
\usepackage{amsmath}
\usepackage{amsthm}
\usepackage[utf8]{inputenc}
\usepackage{pdflscape}
\usepackage{multirow,bigdelim}
\usepackage{hyperref}
\usepackage{multicol}
\title{A note on (locally) presentable categories}
\author{Ambroise LAFONT}
\usepackage{tikz-cd}
\usepackage[all]{xy}

\newcommand{\CC}{\mathsf{C}}
\newcommand{\DD}{\mathsf{D}}
\newcommand{\Cont}[1]{\text{Cont}_{#1}}
\newcommand{\Dir}[1]{\text{Dir}_{#1}}
\newcommand{\Pres}[1]{\text{Pres}_{#1}}
\newcommand{\CwF}{\text{CwF}}
\newcommand{\Lan}{\text{Lan}}
\newcommand{\T}{\mathcal{T}}
\newcommand{\U}{\mathcal{U}}
\newcommand{\id}{\text{id}}
\newcommand{\Set}{\text{Set}}
\newcommand{\im}{\text{Im }}
\newcommand{\bool}{\text{bool}}
\newcommand{\colim}{\mathsf{colim}}
\newtheorem{theorem}{Theorem}[section]
\newtheorem{definition}[theorem]{Definition}
\newtheorem{conjecture}[theorem]{Conjecture}
\newtheorem{proposition}[theorem]{Proposition}
\newtheorem{example}[theorem]{Example}
\newtheorem{corollary}[theorem]{Corollary}
\newtheorem{question}[theorem]{Question}
\newtheorem{remark}[theorem]{Remark}
\newtheorem{lemma}[theorem]{Lemma}
\newtheorem{notation}[theorem]{Notation}
\begin{document}
\maketitle
% This is a report of my visit to Ambrus Kaposi in Budapest from 26th August to
% 30th August 2019, and various thoughts

\tableofcontents

In this document, conjectures are statements that I believe are true, although I did not
have checked them or find a proper reference for them.
The proofs of \cite{adamek_rosicky} are here reformulated, sometimes using (co)ends
(see~\cite{coend} for a demonstration of their usefulness).

A \emph{locally presentable category} (or just \emph{presentable} category) is a
category which is not too big and not too small: it is cocomplete, but is the
generated by a small set of objects.
The main reference for this beast is \cite{adamek_rosicky}.


A morphism between presentable categories is defined as cocontinuous functor in
Anel's and Joyal's Topo-logie.

Presentable categories can be defined as
cocomplete \emph{accessible categories}, or as complete accessible categories
(after nlab, the page on accessible categories and locally presentable categories).

An accessible category is defined as a $\lambda$-accessible category (you don't
need to know what it is at this stage) for some
regular cardinal $\lambda$. This a notion of $\lambda$-presentable category, and
a presentable categories is a $\lambda$-presentable category, for some regular
cardinal $\lambda$. Any $\lambda$-presentable category is also $\mu$-presentable
for any regular cardinal $\mu>\lambda$. This is not the case for accessible
categories! We only know, for them, that given any regular cardinal $\lambda$,
there exists $\kappa>\lambda$ such that any $\lambda$-accessible category is
also $\kappa$-accessible \cite[Theorem 2.11]{adamek_rosicky} (such a condition
is noted $\lambda \triangleleft \kappa$ and $\lambda$ is said sharply smaller,
and it is transitive).
In particular, $\omega\triangleleft \lambda$ for any (infinite) regular cardinal
$\lambda$, and a regular cardinal is always sharply smaller than its successor.
For each set of regular cardinals, there is one which is sharply greater than
all of them. A similar thing can be said of accessible functors.


Presentable categories satisfy nice properties.

% \begin{definition}
%  A functor is said \textbf{to have a rank}  if it has a rank $\lambda$,
%  for some regular cardinal $\lambda$, that is, if it
%   preserves $\lambda$-directed colimits.
%   A functor is said ($\lambda$-)\textbf{accessible} if it has a rank ($\lambda$) and the domain/codomain
%   are ($\lambda$-)accessible.
% \end{definition}
\begin{theorem}
  \cite[1.66]{adamek_rosicky}
  A functor between presentable categories is right adjoint iff it is continuous
  and preserve $\lambda$-directed colimits for some regular cardinal $\lambda$.
  \end{theorem}
  A similar version is true of left adjointness according to nlab (wihtout the
  accessibility condition because it is automatically satisfied).
  In fact, by looking at the proof of \cite[Representation theorem
  1.45]{adamek_rosicky}, it seems that any cocontinous functor out of a
  presentable category and to a copowered category
  has a right adjoint (because any presentable category is of the shape $\Cont\lambda(A)$, and in the proof mentioned above, they say that the cocontinuous induced functor to a cocomplete category has a left adjoint obtained easily using yoneda. I suspect cocompleteness of the target is not necessary here).

  \begin{remark}
	  Anticipating the fact that any presentable category is of the shape $\Cont\lambda(A)$,
	  a right adjoint to a presentable categorie is always a (kind of) nerve functor. This is seen, by postcomposing with the right adjoint inclusion in $\hat{A}$, and using yoneda to get the definition of this right adjoint.
  \end{remark}

  \begin{theorem}[nlab]
   Let $T$  be a monad preserving $\lambda$-directed colimits on a presentable category $C$.
   Then $C^T$ is presentable.
  \end{theorem}
  % Indeed, the right adjoint of a monadic adjunction

\section{Regular cardinals}
\begin{definition}
 A cardinal is regular if it is strictly greater than any smaller union of smaller cardinals.
\end{definition}
Any successor cardinal is regular. The existence of regular uncountable limit
cardinal (called weakly inaccessible cardinal) is not known to be consistent with
ZFC. It is consistent to assume that any uncountable limit cardinal is not regular.

\section{Filtered/directed colimits}
A $\lambda$-filtered category is a category $\CC$ in which any diagram
$\DD\to \CC$ with $\DD$ $\lambda$-small has a cocone. A $\lambda$-directed
category is a poset which induces a $\lambda$-filtered category.

For any $\lambda$-filtered category $\CC$, there is a $\lambda$-directed one
$\DD$ with a cofinal functor $\DD\to\CC$. The existence of a cofinal functor
means that colimits with respect to $\CC$ is the same as colimits with respect
to $\DD$.

That is why we can restrict ourselves to $\lambda$-directed colimits, rather
than the more general $\lambda$-filtered colimits. We can also restrict to
chains for $\omega$-filtered colimits, although there is no cofinality argument
(see~\cite[Example 1.8]{adamek_rosicky}).
% Also, we have this wonderful theorem:
\begin{theorem}
 \cite[Corollary 1.7]{adamek_rosicky}:
 A category has $\omega$-filtered (or $\omega$-directed) colimits iff it has
 colimits of chains. For such categories $K$, a functor of domain $K$ preserve
 $\omega$-filtered colimits iff it preserves colimits of chains.
\end{theorem}


\section{Preliminaries}
By reading \cite{commutationlim} dating back from 2014, it seems that few is
known about the commutation of limits and colimits in $\Set$, except for
particular shapes.

\begin{notation}
  Given a presheaf $F : A \rightarrow \Set$,
  we denote $\int F$  or $\oint^{a:A} F(a)$
  (following \cite{coend})
  the category of elements of $F$.
\end{notation}
\begin{notation}
  We denote the limit of a functor $F:A\rightarrow B$ by $\lim F = \int_{a:A}F(a)$.
  We denote the colimit of a functor $F:A\rightarrow B$ by $\colim F = \int^{a:A}F(a)$.
\end{notation}
\begin{notation}
  The opposite of category $A$ is denoted by $A^o$.
\end{notation}
\begin{definition}
  A \textbf{presheaf} is a functor $A \rightarrow \Set$, where $A$ is a small
  category.
  We denote the category of presheaves $A^o \rightarrow \Set $ by $\hat{A}$.
\end{definition}

\section{Recap on coends}
See \cite{coend} for a compelling view on the usefulness of coends.
Here are some results (some well-known, some less).
\begin{lemma}
 $[C,D](F,G)\simeq \int_c D(F(c),G(c))$.
\end{lemma}
\begin{proposition}[Weighted colimits reduce to conical colimits]
\label{p:weighted-colims}
  Let $W : C^o \rightarrow \Set$ and $F : C\rightarrow D$ be functors, where $D$ has copowers and
  $C$ is small.
  The weighted colimit $\int^c W(c)\times F(c)$ computes as the conical colimit
  $\int^{(c,x):\int W} F(c)$.
\end{proposition}
\begin{corollary}
\label{cor:weighted-colims}
  Here is a list of easy consequences of Proposition~\ref{p:weighted-colims}
  \begin{enumerate}
  \item The weighted limit $\int^c  F(c)^{W(c)}$ computes as the conical colimit
    $\int_{(c,x):\int W} F(c)$.
  \item (coyoneda lemma)
    $\int^c F(c)\times C(c,c')\simeq F(c')$
  \item (yoneda lemma)
    $\int_c F(c)^{C(c',c)}\simeq F(c')$
  \item (coends as colimits) for any functor $H:C\times C^o \rightarrow D$,
      denoting by $y:C \times C^o \rightarrow \Set$ the yoneda bifunctor, we
      have $\int^c H(c,c) \simeq \int^{(c_1\xrightarrow{f}{}c_2):\int y} H(c_1, c_2)$.
      \label{it-coend-colims}
  \item (coend as limits) (the dual of the previous item)
  \end{enumerate}
\end{corollary}
\begin{proof}
  The first statement comes from duality. The yoneda lemma is the dual of the
  coyoneda lemma.
  The coyoneda lemma comes from the fact that $\int C(-,c') = C/c'$. Thus,
  \[
    \int^c F(c)\times C(c,c')
    \simeq 
    \int^{(c\rightarrow c'):C/c'} F(c)
  \]
  Thus, this is a colimit over $C/c'$. But $id_{c'}$ is terminal in $C/c'$, so the
  colimit is trivial and results in $F(c')$.

  For coends as colimits, start from the right hand side:
  \begin{align*}
    \int^{(c_1\xrightarrow{f}{}c_2):\int y} H(c_1, c_2)
    & \simeq
    \int^{c_1,c_2} H(c_1, c_2)\times C(c_2,c_1)
      \text{\quad by Proposition~\ref{p:weighted-colims}.}
      \\
    & \simeq
      \int^{c_1} H(c_1, c_1)
      \text{\quad by the coyoneda lemma.}
    \end{align*}
\end{proof}

Thus, the elimination of weighted colimits (Proposition~\ref{p:weighted-colims}) can be seen as a generalization
of the yoneda lemma.
Interestingly, we can also deduce Proposition~\ref{p:weighted-colims} from
the (co)yoneda lemma and the computation of coends as colimits
(item~\ref{it-coend-colims} in the corollary above): see Lemma~\ref{l:weighted-colims}.


% \begin{lemma}
%   In a distributive complete/cocomplete category (when products commute with colimits),
%   a canonical morphism
%   \[
% \int^{c} \int_{d:D} \sum_{i:I} M_i(c,d) \rightarrow
% \int_d\int^c  \sum_i M_i(c,d) 
%   \]
%   is canonically isomorphic
%   to 
%   \[
%     \sum_{i : \pi_0(D)\rightarrow I}\int^c \int_d M_{i_d}(c,d) \rightarrow
%     \sum_{i : \pi_0(D)\rightarrow I}\int_d\int^c   M_{i_d}(c,d) 
%   \]
% \end{lemma}
% \begin{proof}
%  It comes from the fact that the canonical morphism $\sum_{i\in I}\int_d
%  \rightarrow
%  \sum_{i : \pi_0(D)\rightarrow I}\int_d $  is an isomorphism.

%  More precisely, $\int_d = \prod_{x\in \pi_0(D)}\int_{d:D_x}$, where $D_x$
%  denotes the subcategory of $D$ corresponding to the connected component $x$.
% \end{proof}
\begin{notation}
  \label{n:colim-twisted}
 If $F : \int y \rightarrow \Set$, with $y: C\times C^o \rightarrow \Set$, we
 denote its colimit by $\int^c F(c,c,id_c)$, sometimes even omitting $id_c$ in the expression.
 Thanks to item~\ref{it-coend-colims} of Corollary~\ref{cor:weighted-colims},
 it is consistent with the coend notation
 $\int^c G(c,c)$ because in such $G$ induces by precomposition a functor
 $G':\int y \rightarrow \Set$ and then $\int^c G'(c) = \int^c G(c,c)$.
 \textcolor{red}{Faire une theorie generale des cofins fondees sur cette notation}
\end{notation}
\begin{lemma}
  \label{l:colim-colim}
  Let $F: C\times C^o \rightarrow Cat$ be a functor to small categories with $C$ small and
  $J:\int^{c}F(c,c)\rightarrow D$. Then
  \[
    \int^{(c;a): \int^c F(c,c)} J(c;a) \simeq
    \int^{(c_1 \xrightarrow{f}{} c_2): \int y} \int^{a: F(c_1,c_2)} J(c_1;f(a)).
  \] 
\end{lemma}
\begin{remark}
  With notation~\ref{n:colim-twisted}, this becomes
  \[
    \int^{(c;a): \int^c A(c,c)} J(c;a) \simeq
    \int^c \int^{a: A(c,c)} J(c;a)
  \] 
  Furthermore, by duality,
  \[
    \int_{(c;a): \int^c A(c,c)} J(c;a) \simeq
    \int_c \int_{a: A(c,c)} J(c;a)
  \] 
\end{remark}
\begin{proof}[Proof of Lemma~\ref{l:colim-colim}]
  It relies on the fact (TODO: check) that $[\int^c A(c,c), D]$ is isomorphic to
  $\int_c[A(c,c),D]$.
\end{proof}
\begin{lemma}
  \label{l:compute-grothendieck}
  Let $F:A \rightarrow \Set$  be a presheaf. Then
  \[
    \oint^a F(a) \simeq \int^a F(a)\times a/A
  \]
\end{lemma}
\begin{proof}
  It comes from the coyoneda lemma, the fact that $a/A = \oint^{a'} A(a,a')$ and colimits
  commute with the Grothendieck construction.
\end{proof}
\begin{lemma}
  \label{l:colim-on-elements}
  Let $F : C\rightarrow \Set$ a presheaf and $J:\int F \rightarrow D$ a functor,
  where $D$ has some suitable properties.
  Then,
  \[
    \int^{(c,x): \int F}  J(c,x)\simeq
    \int^{(c_1\xrightarrow{f} c_2):  \int y}
    \sum_{x\in F(c_1)}
    J(c_2,
    f (x))
  \] 
\end{lemma}
\begin{remark}
  With notation~\ref{n:colim-twisted}, this becomes
  \[
    \int^{(c,x): \int F}  J(c,x)\simeq
    \int^{c}
    \sum_{x\in F(c)}
    J(c, x)
  \] 
  Furthermore, by duality,
  \[
    \int_{(c,x): \int F}  J(c,x)\simeq
    \int_{c}
    \prod_{x\in F(c)}
    J(c, x)
  \] 
  \textcolor{red}{TODO: generalise to arbitrary coends (seems necessary in Lemma~\ref{l:gen-lim-dep-pairs})}
\end{remark}
\begin{proof}[Proof of Lemma~\ref{l:colim-on-elements}]
  Let $F : C \rightarrow \Set$. Then (remember that the Grothendieck
  construction preserves colimits):
  \begin{align*}
    \int F = \oint^c F(c) &
                            \simeq
                             \int^c F(c)\times c/A
                            \text{\quad by Lemma~\ref{l:compute-grothendieck}}
                            \end{align*}
%                             \\ & \simeq

%     \\ & \simeq
%                             \simeq  
%                             \oint^c \int^{c'} F(c')\times C(c',c) 
%                             \text{\quad by coyoneda}
%     % \int^{c_1,c_2} C(c_1,c_2)\times F(c_1) \times \oint^c C(c_2, c)
%     \\ & \simeq
%          \oint^c \int^{(c_1\xrightarrow{f} c_2):  \int y} F(c_2) \times  C(c_1,c)
%          \text{\quad by Lemma~\ref{l:comp-coend}}
%     \\
%     & \simeq \int^{(c_1\xrightarrow{f} c_2):  \int y} F(c_2) \times \oint^c C(c_1,c)
%     \text{\quad as $\oint^c$ commutes with $\int^d$}
%     \\
% & \simeq \int^{(c_1\xrightarrow{f} c_2):  \int y} F(c_2) \times c_1/C
%   \end{align*}
  Then, by Lemma~\ref{l:colim-colim},
  \begin{align*}
    \int^{(c,x): \int F}  J(c,x)
    & \simeq
    \int^{(c_1\xrightarrow{f}{} c_2):  \int y}
\int^{(x,c_2 \xrightarrow{u}{} c):F(c_2) \times  c_1/C}
    J(c, u(f(x)))
    \\
    &
      \simeq
      \int^{(c_1\xrightarrow{f}{} c_2):  \int y}
      \sum_{x\in F(c_2)}
      \int^{(c_1 \xrightarrow{u}{} c):   c_1/C}
      J(c, u(f(x)))
  \end{align*}
  But note that $id_{c_1}$ is initial in $c_1/C$. Therefore, for any $K:(c_1/C)^o\rightarrow
  D$, we have that $\colim K\simeq K(id_{c_1})$. This concludes the
  argument.
\end{proof}
\begin{lemma}[Weighted colimits reduce to conical colimits]
\label{l:weighted-colims}
  Here, we state Proposition~\label{p:weighted-colims} and prove it from two of
  its corollaries: the yoneda lemma, and the computation of coends as colimits
  (item~\ref{it-coend-colims} of Propostion~\ref{cor:weighted-colims}).
  \textcolor{red}{TODO: check that the proof does not secretly relies on the conclusion.}
  
  Let $W : C^o \rightarrow \Set$ and $F : C\rightarrow D$ be functors, where $D$ has copowers and
  $C$ is small.
  The weighted colimit $\int^c W(c)\times F(c)$ computes as the conical colimit
  $\int^{(c,x):\int W} F(c)$.
\end{lemma}
\begin{proof}
  \begin{align*}
    \int^{(c,x):\int W} F(c)
    & \simeq
      \int^{(c_1\xrightarrow{f} c_2):  \int y}
      \sum_{x\in W(c_1)}
      F(c_2) \text{\quad by Lemma~\ref{l:colim-on-elements}}
  \end{align*}
  On the other hand,
  \begin{align*}
    \int^c W(c)\times F(c)
    & \simeq
      \int^{(c_1\xrightarrow{f} c_2):  \int y}
      W(c_1)\times
      F(c_2) \text{\quad by item~\ref{it-coend-colims} of Corollary~\ref{cor:weighted-colims}}
  \end{align*}
\end{proof}
% it is sufficient to prove it for $y$-weighted colimits, where $y$ is the yoneda
% embedding $y: C\times C^o \rightarrow \Set$.  
% in this case, this is just the
% fact that a coend is computed as a colimit over the twisted category of arrows
% (which is $\int y$).
\begin{corollary}
  \[
 \int^{(c,x)\in \oint^c F(c)} J(c) \simeq \int^c F(c)\times J(c) 
\]
\end{corollary}
\begin{proof}
  \begin{align*}
    \int^{(c,x)\in \oint^c F(c)} J(c) 
    & \simeq
      \int^{(c_1\xrightarrow{f}{} c_2):  \int y}
      \sum_{x\in F(c_1)} J(c_2)
      \\
    & \simeq
      \int^{c}
      \sum_{x\in F(c)} J(c)
  \end{align*}
\end{proof}
\begin{lemma}[Limits commute with dependent pairs]
  \label{l:lim-dep-pairs}
  The canonical morphism \[
    \int_{d:D} \sum_{x\in K(d)} G(d,x) \rightarrow
    \sum_{\alpha\in \lim K}\int_d G(d,\alpha_d)
  \] 
  is an isomorphism.
\end{lemma}
\begin{remark}
  Recall that the mapping $H \mapsto H'=\sum_{x\in G(-)} H(c,-)$ induces an
  equivalence between $[\int G, \Set]$ and $[C,\Set]/G$.
  The converse functor take a functor $F \xrightarrow{\alpha}{} G$ and maps it
  to the functor $(c,g)\mapsto \alpha^{-1}_c(\{g\})$.
    Then, Lemma~\ref{l:gen-lim-dep-pairs} says that a section of $H'$ is a
    section of $G$ together with a dependent section.
\end{remark}
In fact, we prove a stronger result:
\begin{lemma}
\label{l:gen-lim-dep-pairs}
  Let $F,G : C \rightarrow \Set$ and $H: \int G \rightarrow \Set$ be a functor.
  Then,
  \[
    [C,\Set](F, \sum_{x\in G(-)} H(c,-)) \simeq
    \sum_{\alpha : F \rightarrow G} \int_{(c,x)\in \int F} H(c, \alpha_c(x))
    \]
\end{lemma}
\begin{proof}
  A natural transformation between $F$ and
  $H'=\sum_{x\in G(-)} H(c,-))$ is the same as a natural transformation $\alpha:F
  \rightarrow G$, and a morphism $F \rightarrow H'$ over $G$.
  Through the equivalence with the category $[\int G, \Set]$, this second
  component is equivalently a morphism $[\int G, \Set](\alpha^{-1}(\{-\}), H)$,
  that is, an element of \[
    \int_{c,g}\Set (\alpha^{-1}_c(g),H(c,g))
  \]
  We want it to be isomorphic to:
  \begin{align*}
    \int_{(c,x)\in \int F} H(c, \alpha_c(x))
    & \simeq
      \int_{(c,x)\in \int F} H(c, \alpha_c(x))
  \end{align*}
  \textcolor{red}{TODO: finish}
\end{proof}
\begin{proof} [Proof of Lemma~\ref{l:lim-dep-pairs}]
 Apply Lemma~\ref{l:gen-lim-dep-pairs} with $F = 1$.
\end{proof}

\section{Doctrines}
\textcolor{red}{Il y a un truc a faire avec les foncteurs familiaux!! qui
  preservent presque les limites connectees, mais pas toutes. Pourtant il y a
  toujours une histoire de coproduits, qui est le dual des limites connectees.}
\textcolor{red}{avec l'histoire des sketches, va falloir reflechir a pourquoi on
  peut toujours se ramener a $\mathcal{S}$ l'ensembles des categories $\lambda$-petites}
\begin{definition}
  A \textbf{doctrine} $\mathcal{S}$ (terminology borrowed from \cite{lack_accessible}) is
  a (possibly large) collection of small categories, called shapes.
  We say that a colimit is $\mathcal{S}$-shaped if
  the domain of the diagram lies in $\mathcal{S}$.
\end{definition}
\begin{definition}
 Let $C$ and $D$ be two small categories. The \textbf{pole $Z(C\bot D)$ of $C$ against
   $D$} is the collection of functors $F:C\times D \rightarrow \Set$ such that
 the canonical morphism
 \[
   \int^c \int_d F(c,d)
   \rightarrow  \int_d \int^c F(c,d)
 \]
 is an isomorphism.
 If $Z(C\bot D) = [C\times D, \Set]$, then we say that the pole is
 \textbf{maximal} and that $C$-colimits commute with $D$-limits.
\end{definition}
\begin{remark}
  $F : C \times D \rightarrow \Set$ is in $Z(C\bot D)$ is equivalent to the
  preservation by $\Lan_y 1 : [C,\Set] \rightarrow \Set$ of the limit of the diagram
  $D \rightarrow [C, \Set]$ induced by $F$.
  Consequently, $C$-colimits commute with $D$-limits if $\Lan_y 1$ is $D$-continuous.
\end{remark}
\begin{remark}
  \begin{align*}
    \Lan_yF(G)=\Lan_yG(F) &&\\ 
    \Lan_{y_{\int F}} 1(G) = \Lan_y F(G) &&\text{\qquad if $G$ factors through the
      forgetful functor $\int F \rightarrow C$.}
    \end{align*}
    The last equality is the content of elimination of weights.
    In fact, $\Lan_y F(G)$ is the $G$-weighted colimit of $F$ (and reciprocally).
\end{remark}
\begin{definition}
 Let $F : C \rightarrow \Set$ be a presheaf. A \textbf{dependent $F$-weighted
   colimit} is a colimit of a functor $G: (\int F)^o \rightarrow F$.
\end{definition}
\begin{remark}
 Thanks to the reduction of weights, $F$-weighted colimits are particular cases
 of dependent ones.
\end{remark}
% \begin{definition}
%   Given two small categories $C$ and $D$, we say that $C$-colimits commute
%   over $D$-limits if $Z(C\bot D) = [C\times D, \Set]$, that is, for any functor $F:C\times D \rightarrow \Set$, the
%   canonical morphism
%   \[
%     \int^c \int_d F(c,d)
%     \rightarrow  \int_d \int^c F(c,d)
%   \]
%   is an isomorphism.
% \end{definition}
\begin{definition}
  We denote $\mathcal{S}^\bot$ the doctrine such that
  $\mathcal{S}^\bot$-shaped colimits commute with 
  $\mathcal{S}$-shaped limits in $\Set$.
\end{definition}
% \begin{definition}
%   A \textbf{pole of a doctrine $\mathcal{S}$} is possibly large collection of
%   functor $C\times D \rightarrow $
% \end{definition}
\begin{remark}
  Considering~\cite{lack_accessible, commutationlim}, little is known about the
  general commutation of limits with colimits, except for specific doctrines.
  Typically, we consider small doctrines $\mathcal{S}$ and then
  $\mathcal{S}^\bot$ is large.
\end{remark}
\begin{remark}
$\mathcal{S}^\bot$ is closed under finite products.
\end{remark}
% \begin{definition}
%   A doctrine $\mathcal{S}$ is 
%   \textbf{delightful} if, given any presheaf $F:C\rightarrow \Set$
%   $D \in \mathcal{S}$, if functors of the shape $(\int F)^o \times D \rightarrow C^o
%   \times D \rightarrow \Set$ are in $Z((\int F)^o \bot D)$, then 
%   $Z((\int F)^o \bot D)$ is maximal.
% \end{definition}
% \begin{remark}
%   In other words, a doctrine $\mathcal{S}$ is deligthful if,
%   for $(\int F)^o$ to be in $\mathcal{S}^\bot$, it is enough that
%   for any functor
%   $H:(\int F)^o \times D \rightarrow \Set$ with $D \in \mathcal{S}$, the 
%   canonical morphism
%   \[
%     \int^{(c,x): \int F} \int_d H(c,d) \rightarrow
%     \int_d \int^{(c,x): \int F}  H(c,d) 
%   \]
%   is an isomorphism.
% \end{remark}
% \begin{remark}
%   There is an equivalent definition of delightful doctrines which only
%   talks about poles of $F = y : C^o \times C \rightarrow \Set$,
%   because a coend is computed as a colimit over a twisted arrow category
%   (Item~\ref{it-coend-colims} of Corollary~\ref{cor:weighted-colims}), which is
%   a category of the shape $(\int y)^o$.
% \end{remark}
% \begin{lemma}
%   A doctrine $\mathcal{S}$ is 
%   {delightful} if and only if, given any category $C$,
%   for $(\int y)^o$ to be in $\mathcal{S}^\bot$, where $y : C\times C^o
%   \rightarrow \Set$ is the yoneda embedding, it is enough that
%   for any functor
%   $H:(\int y)^o \times D \rightarrow \Set$, the 
%   canonical morphism
%   \[
%     \int^{(c,x): \int y} \int_d H(c,d) \rightarrow
%     \int_d \int^{(c,x): \int y}  H(c,d) 
%   \]
%   is an isomorphism.
% \end{lemma}
% \begin{proof}
%   Because a coend is computed as a colimit on the twisted category, which is
%   a category of the shape $\int y$.
% \end{proof}
% \begin{definition}
%  Let $\mathcal{S}$ be a doctrine. A presheaf $F:A \rightarrow \Set$ is
%  called \textbf{$\mathcal{S}$-flat} if its left Kan extension 
%  preserve $\mathcal{S}$-shaped limits of representable presheaves.
% \end{definition}
% \begin{remark}
%   If a presheaf $F : A \rightarrow \Set$ is $\mathcal{S}$-flat then, it
%   preserves $\mathcal{S}$-shaped limits. The converse is true if
%   $A$ is $\mathcal{S}$-complete.
%   \end{remark}
  \begin{lemma}
    \label{l:flat-implications}
    Let $y : C^o \times C \rightarrow \Set$ be the yoneda functor, with $C$ small.
We have an easy chain of implications, for a presheaf $F : C \rightarrow \Set$:
\begin{enumerate}
\item $(\int F)^o \in \mathcal{S}^{\bot}$, that is,
  dependent $F$-weighted colimits commute with $\mathcal{S}$-limits
  (we then say that $F$ is \textbf{$\mathcal{S}$-flat}) \label{it-el-presheaves};
\item $F$ is a $\mathcal{S}^{\bot}$-colimit of representable presheaves
  (because it is a $F$-weighted colimit of representable presehaves);
\item $\Lan_yF$ preserves any $\mathcal{S}$-shaped limits
  ($F$-weighted colimits commute with $\mathcal{S}$-limits)
  \label{it-lan-preserves};
\item $\Lan_yF$ preserves any $\mathcal{S}$-shaped limits of representable presheaves;
  ($F$-weighted colimits commute with $\mathcal{S}$-limits of representable presheaves)
  \label{it-lan-preserves-rep}
  \item $F$ preserves any $S$-shaped limits
    ($F$-weighted colimits commute with $\mathcal{S}$-limits of representable
    presheaves which already exist in $C$).
    \label{it-F-preserves}
\end{enumerate}
Moreover, if $C$ is $\mathcal{S}$-complete, then
item~\ref{it-F-preserves} entails item~\ref{it-lan-preserves-rep}.
\end{lemma}
\begin{proof}
 Straightforward using the coend formula for the left Kan extension. 
\end{proof}
\begin{definition}
  Let $\mathcal{S}$ be a doctrine.
  It is \textbf{deligthful} if 
  item~\ref{it-lan-preserves} implies item~\ref{it-el-presheaves} 
  in Lemma~\ref{l:flat-implications}, i.e. for any presheaf $F$, if $F$-weighted
  colimits commute with $\mathcal{S}$-limits, then so do any dependent $F$-weighted colimits.
  As a consequence, the first \ref{it-lan-preserves} items
  are logically equivalent.
\end{definition}
That is why a delightful doctrine is delightful.
\cite{lack_accessible} proposes the subclass of sound doctrines.
% (proved delightful by Lemma~\ref{l:sound-delightful}).
\begin{definition}
  A doctrine $\mathcal{S}$ is \textbf{sound for a collection of presheaves $F :
    C \rightarrow \Set$}
  % if
  % A doctrine $\mathcal{S}$ is \textbf{sound}, if
  if
  item~\ref{it-lan-preserves-rep} in Lemma~\ref{l:flat-implications} entails
  item~\ref{it-el-presheaves}, so that
  all theses items up to item~\ref{it-el-presheaves} are
  equivalent %~\cite[Theorem 2.4]{lack_accessible}
  (and thus it is delightful).
  In other words, if $F$-weighted colimits commute with $\mathcal{S}$-limits of
  representable presheaves, then do so any dependent $F$-weighted colimits.

  A doctrine is said \textbf{sound} if it is sound for the total collection of presheaves.
  % A doctrine $\mathcal{S}$ is \textbf{sound} \cite[Definition 2.2]{lack_accessible}
  % if, given any small categories $C$, % with $D \in \mathcal{S}$,
  % if $C$-colimits commute $\mathcal{S}$-limits of representable presheaves, then $C$
  % commutes with $\mathcal{S}$-limits.
  % if the pole $Z(C \bot D)$ contains all the functors of the shape $C(S -,=)$, with $S :
  % D^o \rightarrow C$, then it is maximal.
  % % then $C \in \mathcal{S}^\bot$.
  % % $D \in \mathcal{S}$, if functors of the shape $(\int F)^o \times D \rightarrow C^o
  % \times D \rightarrow \Set$ are in $Z(\int F \bot D)$, then 
  % in fact $Z(\int F \bot D) = [\int F\times D, \Set]$.
  % if, for a small category $C$ to be in $\mathcal{S}^\bot$,
  % given any presheaf $F:C\rightarrow \Set$,
  % for $(\int F)^o$ to be in $\mathcal{S}^\bot$,
  % it is enough that
  % that for any functor
  % $H:D \rightarrow C $ with $D \in \mathcal{S}$, the 
  % canonical morphism
  % \[
  %   \int^{c: C} \int_d D(c, H d) \rightarrow
  %   \int_d \int^{c}  D(c, H d) 
  % \]
  % is an isomorphism.
\end{definition}
\begin{example}
  Here are some examples \cite[Example 2.3]{lack_accessible} of sound doctrines:
  \begin{itemize}
  \item (easy)
    $\mathcal{S} = \{ \lambda\text{-small categories} \}$, and then
    $\mathcal{S}^\bot$ is the collection of $\lambda$-filtered categories;
  \item (non trivial) the doctrine of finite discrete categories;
  \item 
    $\mathcal{S} = \{ \text{finite connected categories} \}$, and then
    $\mathcal{S}^\bot$ is the collection of sifted categories;
  \item (non-examples) the doctrine of countable products (i.e. countable
    discrete categories) and the doctrine of pullbacks are not sound.
  \end{itemize}
\end{example}
\begin{remark}
 If a doctrine $\mathcal{S}$ is sound on terminal presheaves, % then \cite[Definition 2.2]{lack_accessible}
 then, given any small category $C$, % with $D \in \mathcal{S}$,
 if $C$-colimits commute $\mathcal{S}$-limits of representable presheaves, then $C$
 commutes with $\mathcal{S}$-limits.

  \label{r:sound-connected}
  Note that by yoneda, the codomain of the canonical 
  \[
    \int^{c: C} \int_d C(S d, c) \rightarrow
    \int_d \int^{c}  C(S d, c) 
  \]
  is a singleton set.
  As a colimit consists in taking the connected components of the category of elements,
  another way to state soundness is that the category of cocones of any functor
  $S : D^o \rightarrow C$ is connected.
  As a consequence \cite[Proposition 2.5]{lack_accessible}, any $\mathcal{S}^o$-cocomplete contegory $C$ is in
  $\mathcal{S}^\bot$ (could also be deduced directly from $\int_d C( S d, c)
  \simeq C(\int^d S d, c)$).
\end{remark}
% \begin{question}
%  Why not require that the category of cones is connected? 
% \end{question}
In fact, any doctrine satisfying this property is sound.

\begin{lemma}
  \label{l:sound-delightful}
  If a doctrine $\mathcal{S}$ is sound on terminal presheaves, then
%   item~\ref{it-lan-preserves-rep} in Lemma~\ref{l:flat-implications} entails
% item~\ref{it-el-presheaves}, so that
%   all theses items up to item~\ref{it-el-presheaves} are
  it is sound~\cite[Theorem 2.4]{lack_accessible}.
\end{lemma}
\begin{proof}
  So we are given a presheaf $F:C\rightarrow \Set$ such that 
  for any functor
  $H:(\int F)^o \times D \rightarrow \Set$, the 
  canonical morphism
  \[
    \int^{(c,x): \int F} \int_d H(c,d) \rightarrow
    \int_d \int^{(c,x): \int F}  H(c,d) 
  \]
  is an isomorphism.
  Now we would like to show that $(\int F)^o \in \mathcal{S}^\bot$. As
  $\mathcal{S}$ is sound, it is enough to show that given any functor
  $H:D \rightarrow (\int F)^oC $ with $D \in \mathcal{S}$, the 
  canonical morphism
  \[
    \int^{f: (\int F)^o} \int_d \smallint F (f, H d) \rightarrow
    \int_d \int^{f}  \smallint F(f, H d) 
  \]
  is an isomorphism.
  
  Let us rephrase this statement: by universal property
of $\int F$ as the comma category $1/F$,
  a functor 
  $H:D \rightarrow (\int F)^o $ is equivalently given by a functor
  $G : D \rightarrow C^o$ together with an element $\alpha \in \lim FG$.
  We then want that
  \[
    \int^{(c,x): \int F} \int_d  \sum_{f : C(c,G d)} (f(x)=\alpha_d) \rightarrow
    \int_d \int^{(c,x)}  \sum_{f : C(c,G d)} (f(x)=\alpha_d) 
  \]
  is an isomorphism.

  We would like to apply our hypothesis on $\int F$, but we need first to remove the
  the dependency on $x$. The trick is to show take the coproduct of these
  morphisms over all possible $\alpha\in \lim GH$ is an isomorphism.

  By the fact that limits commute with dependent pairs in $\Set$
  (Lemma~\ref{l:lim-dep-pairs}),
  the domain of the resulting coproduct morphism is canonically isomorphic to
\[
  \int^{(c,x): \int F} \int_d  \sum_{x' \in FGd}\sum_{f : C(c,G d)} (f(x)=x')
  \simeq
  \int^{(c,x): \int F} \int_d  C(c,G d).
  \]
  By a similar argument, the codomain of the resulting morphism is canonically
  isomorphic to
  \[
    \int_d \int^{(c,x): \int F}   C(c,G d).
  \]
Now we can apply our hypothesis on $\int F$, which concludes the argument.

  
\end{proof}
\begin{remark}
  In fact, Lemma~\ref{l:sound-delightuful} can be taken as a definition of sound
  doctrines: if 
  item~\ref{it-lan-preserves-rep} in Lemma~\ref{l:flat-implications} entails
  item~\ref{it-el-presheaves}, then the doctrine is sound (consider the case of
  $F$ as the terminal functor).
  % By inspecting the proof above, we see that for sound doctrines,
  % item~\ref{it-lan-preserves-rep} implies the first one in
  % Lemma~\ref{l:flat-implications}. 
  % As a consequence, the first \ref{it-lan-preserves-rep} items
  % are logically equivalent.
\end{remark}
% This leads to the following definition:
% \begin{definition}
%   A doctrine $\mathcal{S}$ is 
%   \textbf{sound} if, given any presheaf $F:C\rightarrow \Set$
%   $D \in \mathcal{S}$,
%   if the pole $Z((\int F)^o \bot D)$ contains all the functors of the shape
%    $(\int F)^o \times D \rightarrow C^o
%   \times D \xrightarrow{C(-, S =)}{} \Set$ for $S : D^o \rightarrow C$, then it is maximal.
%   % if functors of the shape $(\int F)^o \times D \rightarrow C^o
%   % \times D \rightarrow \Set$ are in $Z((\int F)^o \bot D)$,
% \end{definition}
% \begin{lemma}
%   We have an easy chain of implications:
%   \begin{enumerate}
%   \item $\mathcal{S}$ is sound;
%   \item $\mathcal{S}$ is sound;
%   \item $\mathcal{S}$ is delightful.
%   \end{enumerate}
% \end{lemma}
% \begin{example}
%  \cite[Remark 2.6]{lack_accessible} provides a simple example of a doctrine
%  which is not sound, but such that item~\ref{it-lan-preserves-rep} entails
%  item~\ref{it-lan-preserves} in case $C$ is $\mathcal{S}$-complete, in Lemma~\ref{l:flat-implications}: $\mathcal{S} = \{\leftarrow
%  \rightarrow, \cdot\}.$
% \end{example}
% \begin{question}
%   Is it an example of a sound doctrine which is not sound? 
%   \textcolor{red}{N'est ce pas en contradiction avec \cite[Remark 2.7]{lack_accessible}}
% \end{question}
% \begin{lemma}
%  \cite[Remark 2.7]{lack_accessible} says that sound doctrine are exactly those
%  for which item~\ref{it-lan-preserves-rep} entails
%  item~\ref{it-lan-preserves}, even when $C$ is not $\mathcal{S}$-complete!
% \end{lemma}
% \begin{remark}
%   \label{r:strong-delight-car}
%   A doctrine is sound exactly when
%   item~\ref{it-lan-preserves-rep} in Lemma~\ref{l:flat-implications} entails the first one.

%   Thus, in this case, given a small category $C$ with $\mathcal{S}$-shaped limits, a presheaf
%   $C \rightarrow \Set$ preserves $\mathcal{S}$-shaped limits if and only it is a
%   $\mathcal{S}^\bot$-shaped colimit of representable presheaves.
% \end{remark}
\begin{conjecture}
  Let $C$ be a small category and
  $\mathcal{S}$ be a doctrine.
  Let
  \begin{itemize}
  \item $U_\mathcal{S}$ be the forgetful 2-functor from categories with $\mathcal{S}$-shaped colimits
    and functors preserving them to the category of small categories.
    \item $V_\mathcal{S}$ be
     be the forgetful 2-functor from cocomplete categories and cocontinuous
     functors to the 2-category of small categories and functors preserving $\mathcal{S}$-shaped
colimits;
\item $\Dir{\mathcal{S}}(C)$ the full subcategory of $\hat{C}$ consisting of
  $\mathcal{S}$-shaped colimits of representable presheaves;
\item $\Cont{\mathcal{S}}(C)$ the full subcategory of $\hat{C}$ consisting of
  presheaves preserving $\mathcal{S}$-shaped colimits.
  \end{itemize}
  Then,
  \begin{itemize}
  \item 
    the yoneda embedding exhibits $\Dir{\mathcal{S}}(C)$ as the weak free
    object for $U_\mathcal{S}$
    (weak in the sense that
    the universal morphism is only up to iso);
    \item
      if $\mathcal{S}$ is a small set, then 
      the yoneda embedding exhibits  $\Cont{\mathcal{S}}(C)$ as the
      weak free object with respect to $V_\mathcal{S}$ (weak in the sense that
      the universal morphism is only up to iso).
  \end{itemize}
\end{conjecture}
This is a conjecture but I will give arguments.
% as it generalizes the case of Example~\ref{ex:S-lambda-small} which is
% treated in \cite{adamek_rosicky}.
\begin{question}
What happens to $\Cont{\mathcal{S}}(C)$ if $\mathcal{S}$ is not small?
\end{question}
\begin{remark}
  $\Dir{\mathcal{S}^\bot}(C)$  is a full subcategory of $\Cont{\mathcal{S}}(C)$.
\end{remark}
\begin{lemma}
  If $\mathcal{S}$ is sound and $C$ is $\mathcal{S}$-complete, then
  $\Dir{\mathcal{S}^\bot}(C)$ is equivalent to $\Cont{\mathcal{S}}(C)$.
\end{lemma}
\begin{proof}
  By Remark~\ref{r:strong-delight-car}.
\end{proof}
\begin{conjecture}
  If $\mathcal{S}$ is sound and
  $\Dir{\mathcal{S}^\bot}(C)$ is equivalent to $\Cont{\mathcal{S}}(C)$, then
  $C$ is $\mathcal{S}$-complete.
\end{conjecture}
  The universal property of $\Dir{\mathcal{S}}(C)$ should be easy to prove.
  We will thus focus on the universal property of $\Cont{\mathcal{S}}(C)$.

  \begin{lemma}
    The yoneda embedding $y : C \rightarrow \Cont{\mathcal{S}}(C)$ preserves
    $\mathcal{S}$-shaped colimits.
  \end{lemma}
  \begin{proof}
    Let $K:D\rightarrow C$ be a $\mathcal{S}$-shaped diagram with an existing colimit in
    $C$. Let us check that $y\colim K$ has the required universal property, that
    is, given any presheaf $F$ in $\Cont{\mathcal{S}}(C)$, we have 
    $\hat{C}(y \colim K, F)$ is canonically isomorphic to $[D, \hat{C}](y K,
    F)\simeq \lim FK$.
    But $\hat{C}(y \colim K, F)$ is canonically isomorphic to $F\colim K$, which
    is canonically isomorphic to $\lim FK$.
  \end{proof}
  First, some interesting (although not necessary lemma):
  \begin{lemma}
    Any cocontinuous functor $L:\Cont{\mathcal{S}}(C)\rightarrow B$,
    with $B$ locally small,
    has a right adjoint.
  \end{lemma}
  \begin{proof}
   If this is true, then the right adjoint is given by $R(b)(x) = \hat{A}(y x,
   R(b)) = B(Ly x, b)$. Here, $R$ clearly defines a functor from $B$ to $\hat{A}$.
   We check that any $R(b)$ preserves $\mathcal{S}$-shaped colimits: let $\colim K$ be a 
   $\mathcal{S}$-shaped colimit in $A$. Then, $R(b)(\colim K) = B(Ly\colim K,b)=
   B(\colim LyK, b) = \lim B(yK,b)=\lim R(b)K$.
   \end{proof}
   \begin{remark}
     This also follows from the adjoint functor theorem stating that any
     cocontinuous functor out of a total category (such as presentable
     categories) is a left adjoint.
   \end{remark}
  \begin{remark}
   This provides examples of a left adjoint (the left Kan extension) which the composition of a right
   adjoint (the embedding in the total presheaf category) and a left adjoint
   (the left Kan extension).
  \end{remark}
  Another noticeable fact:
  \begin{lemma}
    $\Cont{\mathcal{S}}(C)$ is complete.
  \end{lemma}
  \begin{proof}
   Because limits commute with limits.
  \end{proof}
  But in fact, everything follows from the following lemma:
  \begin{lemma}
    If $\mathcal{S}$ is small, then
    $\Cont{\mathcal{S}}(C)$ is a reflective subcategory of $\hat{C}$.
  \end{lemma}
  \begin{proof}
    We apply Conjecture~\ref{conj:fibrant-reflective}.
    We only have to devise an orthogonal factorization system on $\hat{C}$ such that a
    presheaf is fibrant if and only it preserves $\mathcal{S}$-shaped colimits.
    But, observe that a presheaf $F:C^o \rightarrow \Set$  preserves a colimit $\int^i D(i)$ if and
    only if $F\rightarrow 1$ is right orthogonal to $m_D:y_{\int^i D(i)}\rightarrow \int_i y_{D(i)}  $.
    Thus we consider the orthogonal factorization system cofibrantly generated
    by the collection of such morphisms $m_D$. Note that this is a small set
    because $C$ is small and $\mathcal{S}$ also is.
  \end{proof}
  \begin{corollary}
    $\Cont{\mathcal{S}}(C)$ is cocomplete, by Lemma~\ref{l:reflect-cocomplete}.
  \end{corollary}
  \begin{remark}
    $\mathcal{S}^\bot$-shaped colimits in $\Cont{\mathcal{S}}(C)$ are computed pointwise.
  \end{remark}
  \begin{lemma}
    If $\mathcal{S}$ is closed under binary products (for example, $\mathcal{S}
    = {\mathcal{S}'}^\bot$),
    $\Dir{\mathcal{S}}(C)$ has $\mathcal{S}$-colimits, which are
    computed pointwise.
  \end{lemma}
  \begin{lemma}
    Given a functor $F:C \rightarrow B$ preserving $\mathcal{S}$-shaped
    colimits, with $B$ cocomplete, then the left Kan extension of $F$ along the
    yoneda embedding in $\Cont{\mathcal{S}}(C)$ is the unique cocontinuous
    functor (up to iso) yielding $F$ when precomposing with the yoneda embedding.
  \end{lemma}
  \begin{proof}
    That it is cocontinuous comes from Lemma~\ref{l:cocontinuous-lkan}. That it
    is unique up to iso follows from Lemma~\ref{l:unique-cocontinuous-lkan}.
  \end{proof}


  \begin{definition}
    \label{d:presentable-objs}
    Let $\mathcal{C}$ be a locally small category.
    Let $\Pres{\mathcal{S}}(\mathcal{C})$ be the full subcategory of objects $c$ of $\mathcal{C}$ such that 
    $y_c : \mathcal{C} \rightarrow \Set$ is in $\Cont{\mathcal{S}}(\mathcal{C}^o)$, that is,
    $\mathcal{C}(c,-)$ preserves $\mathcal{S}$-shaped colimits.
    In other words, $\Pres{\mathcal{S}}(\mathcal{C})$ is the pullback (beware:
    non locally small categories involved)
    \[
      \xymatrix{
        \Pres{\mathcal{S}}(\mathcal{C}) \ar[r]\ar[d]&
        \mathcal{C}
        \ar[d]
        \\
        \Cont{\mathcal{S}}(\mathcal{C}^o) \ar[r]
        &
        [\mathcal{C}, \Set]
      }
    \]
  \end{definition}
  \begin{conjecture}
    $\Cont{\mathcal{S}}(\mathcal{C}) \rightarrow \hat{\mathcal{C}}$  and $\Dir{\mathcal{S}}(\mathcal{C})$ are
    isofibrations, and thus the pullback
    in Definition~\ref{d:presentable-objs} is also a weak pullback.
  \end{conjecture}
  \begin{lemma}
    \( \Pres{\mathcal{S}^\bot}(\mathcal{C}) \)
    is closed under any $\mathcal{S}$-colimits
    that exists in $\mathcal{C}$. In particular, if
    $\mathcal{C}$ is cocomplete (for example $\mathcal{C} =
    \Cont{\mathcal{S}}(C)$ with $\mathcal{S}$ and $C$ small),
    then $\Pres{\mathcal{S}^\bot}(\mathcal{C})$ has
    $\mathcal{S}$-colimits.
  \end{lemma}
  \begin{conjecture}
    % For $\mathcal{S}$ and $C$ small,
    $\Pres{\mathcal{S}^\bot}(\Cont{\mathcal{S}}(C))$ is the 
    closure of $C$ under $\mathcal{S}$-colimits in $\Cont{\mathcal{S}}(C)$
    (which may not always exist, but do if $\mathcal{S}$ and $C$ are small).
  \end{conjecture}
  \begin{question}
   What about $\Pres{\mathcal{S}^\bot}(\Dir{\mathcal{S}}(C))$?
  \end{question}
  \begin{corollary}
    If $\mathcal{S}$  and $C$ are small, and $C$ is $\mathcal{S}$-cocomplete, then
    $\Pres{\mathcal{S}^\bot}(\Cont{\mathcal{S}}(C))\simeq C$.
    Thus in this case, a presheaf in $\Pres{\mathcal{S}^\bot}(\Cont{\mathcal{S}}(C))$ preserves
    in fact any colimit, as it is a representable presheaf.
  \end{corollary}
  \begin{question}
    Which kind of free cocompletion of $C$ is
    $\Pres{\mathcal{S}^\bot}(\Cont{\mathcal{S}}(C))$?
    % ? Not the usual free one of course
  \end{question}
  \begin{conjecture}
    \begin{enumerate}
    \item 
      $\Dir{\mathcal{S}} \circ \Pres{\mathcal{S}} \circ \Dir{\mathcal{S}} \simeq \Dir{\mathcal{S}} $
      \item
      $\Cont{\mathcal{S}} \circ \Pres{\mathcal{S}^\bot} \circ \Cont{\mathcal{S}} \simeq \Cont{\mathcal{S}} $
    % \item if $C$ has $\mathcal{S}$-shaped colimits, then
    %   $\Pres{\mathcal{S}^\bot}(\Cont{\mathcal{S}}(C))\simeq C$ (and thus
    %   $\Pres{\mathcal{S}^\bot}(\Dir{\mathcal{S}^\bot}(C))\simeq C$).
      % $\Cont{\mathcal{S}}(C) \simeq \Dir{\mathcal{S}^\bot}(C)$
    % \item Given a functor $F:C\rightarrow \Set$, the opposite of its category of
    %   elements $\int F$ is in $\mathcal{S}^\bot$ if and only if its left Kan
    %   extension preserves $\mathcal{S}$-shaped limits.
    \end{enumerate}
  \end{conjecture}
  \begin{proof}
  \end{proof}
  \begin{definition}
    A $\mathcal{S}$-presentable category is a category $\mathcal{C}$ such that any object
    is a $\mathcal{S}^\bot$-colimit of objects of $\Pres{\mathcal{S}^\bot}(\mathcal{C})$, and
    $\Pres{\mathcal{S}^\bot}(\mathcal{C})$ is essentially small.
  \end{definition}
  \begin{conjecture}
    Any category is $\mathcal{S}$-presentable if and only if it is equivalent to some
    $\Cont{\mathcal{S}}(C)$ with $C$ small.
  \end{conjecture}
  \begin{conjecture}[Gabriel-Ulmer] 
    There is a 2-equivalence between $\mathcal{S}$-complete categories and 
    $\mathcal{S}$-presentable category (through $\Pres{\mathcal{S}^\bot}$ and $\Cont{\mathcal{S}}$).
  \end{conjecture}
  \begin{conjecture}
    If $\mathcal{C}$ is $\mathcal{S}$-presentable and $\mathcal{S}$ is sound and
    closed under opposite,
    then $\Pres{\mathcal{S}^\bot}(\mathcal{C})$ is
    $\mathcal{S}^o$-cocomplete, and thus is in $\mathcal{S}^\bot$, by Remark~\ref{r:sound-connected}.
  \end{conjecture}
  \section{Stuff about presentable categories}
  \begin{lemma}
    Given a $\lambda$-presentable category $\mathcal{C}$, there exists
    $\mu>\lambda$ such that $\Pres\mu(\mathcal{C})$ is stable per subobjects
  \end{lemma}
  \begin{proof}
   Take the closure of $\Pres\lambda(\mathcal{C})$ under subobjects (which is
   still a set). As any
   object is compact in a presentable category, we can take $\mu$ to be such
   that any object in this closure is $\mu$-compact.
  \end{proof}
  \begin{example}
    To find an example of a $\Pres\lambda(\mathcal{C})$ not closed under
    subobjects, it is enough to find a small category $C$ such that some
    subobject of a representable presheaf (that is, some sieve) is not a finite
    colimit of representable presheaves. For example, consider the chain
    category with a terminal object adjoined, and consider the functor to set
    mapping any object to 1, except the terminal object which is mapped to 0.
    This is a sieve, but, it is not a finite colimit of representable presheaves.
  \end{example}
  \begin{conjecture}
    If $C$ is a small category and $\mathcal{C}$ is $\lambda$-presentable, then
    so is $[C, \mathcal{C}]$.
  \end{conjecture}
   
\section{Presentable categories (old new attempt)}

Let $\lambda$ be a regular cardinal.
\begin{definition}
  A \textbf{$\lambda$-presentable object} in a category $C$ is an object such
  that its yoneda embedding preserves $\lambda$-directed colimits.
\end{definition}
It says no more that any morphism from such an object to a $\lambda$-directed
colimit factors (essentially uniquely) through some object of the colimiting cocone.
\begin{lemma}
  A $\lambda$-presentable object is also $\kappa$-presentable, for $\kappa > \lambda$.
\end{lemma}
\begin{proof}
  Obvious, because a $\kappa$-filtered category is also $\lambda$-filtered.
\end{proof}
\begin{definition}
  \label{d:accessible}
  A $\lambda$-accessible category is a category closed
  under $\lambda$-directed colimits and there is a small set $A$ of
  $\lambda$-presentable objects such that every object is a $\lambda$-directed
  colimit of objects of $A$.
\end{definition}
\begin{lemma}
  Consider the full subcategory $\Pres\lambda(C)$ of $\lambda$-presentable
  objects of a category $C$. If $C$ is locally $\lambda$-accessible,
  $\Pres\lambda(C)$ is essentially small (in the following we will reason as if
  it were small), and every object in $C$ is a $\lambda$-directed colimits of objects
  of $\Pres\lambda(C)$.
\end{lemma}
\begin{definition}
  A $\lambda$-presentable category is a $\lambda$-accessible category
  which is cocomplete.
\end{definition}
\begin{proposition}
  \label{d:pres-strong-gen}
  A category is $\lambda$-presentable category if and only if it is a
  cocomplete category with a strong generator formed of $\lambda$-presentable objects.
\end{proposition}
To go from the strong generator to a proper set of generating $\lambda$-presentable objects
above: take $\lambda$-small colimits of objects in the strong generator.
\begin{conjecture}
 Given a $\lambda$-presentable $C$ category with a strong generator $A$ of
 $\lambda$-presentable objects,
 \begin{itemize}
   \item $\Pres\lambda(C)$ is closed under $\lambda$-small colimits;
 \item 
   $\Pres\lambda(C)$ is equivalent to the
   full subcategory of $C$ consisting of $\lambda$-small colimits of elements $A$.
 \end{itemize}
\end{conjecture}
\begin{proof}
  I haven't tried to prove it.
  \end{proof}
\begin{example}
  Any presheaf category is $\omega$-presentable: consider the strong generator consisting
  of representable presheaves.
\end{example}
\begin{lemma}
  A locally $\lambda$-presentable category is also locally $\kappa$-presentable, for $\kappa > \lambda$.
\end{lemma}
\begin{proof}
  We consider the alternative characterisation of locally presentable categories
  of Proposition~\ref{d:pres-strong-gen}.
  The set of ''presenting'' $\kappa$-presentable objects is induced by noticing that
  $\lambda$-presentable objects are also $\kappa$-presentable, and form a strong
  generator. Then, we take $\kappa$-small colimits of them.
\end{proof}

There is an alternative definition of presentable categories as free cocompletions.
\begin{definition}
 Let $U_{\Cont\lambda}$ denotes the forgetful functor from cocomplete categories
 and cocontinuous functors
 to small categories and $\lambda$-cocontinous functors between them (that is,
 functors preserving $\lambda$-small colimits).  In particular, the codomain of $U_{\Cont{0}}$
 is the usual category of small categories.
 Let $\hat{A}$ denotes the presheaf category $[A^o, \Set]$, for a small category $A$.
\end{definition}
In the following, $A$ denotes a small category.
\begin{lemma} The yoneda embedding
$y:A \rightarrow \hat{A}$ exhibits the free object with respect to
$U_{\Cont{0}}$.
The unique cocontinuous functor induced by the universal property is computed as a left Kan
extension.
\label{l:free-cocomp}
  \end{lemma}
  \begin{proof}
    Let $G:A \rightarrow B$ where $B$ is a cocomplete category.
    This induces a functor $Lan_y(G) : \hat{A} \rightarrow B$ which has a right
    adjoint (as explained later, I think this is true even if $B$ were not cocomplete). It is
    cocontinuous and unique because any presheaf is a colimit of representables ones, whose
    image is fixed by $G$.
  \end{proof}
  \begin{lemma}
    \label{l:cont-free}
    (\cite[Proposition 1.45]{adamek_rosicky})
    Let $\Cont\lambda(A)$ be the full subcategory of $\hat{A}$ consisting of
presheaves preserving $\lambda$-small colimits of $A$.
    The yoneda embedding
    $y:A \rightarrow \Cont\lambda{A}$ is the free object of $A$ with respect to $U_{\Cont{\lambda}}$.
  \end{lemma}
   See the reference for the proof. The only thing which is not obvious is the
   fact that $\Cont\lambda{A}$ is cocomplete. This relies on the orthogonal
   reflection section of \cite{adamek_rosicky}.
   We recast this section in \S\ref{ss:ortho} in terms of orthogonal
   factorization systems which look more standard to us. Hopefully, the results
   involved in \S\ref{ss:ortho} could be inserted at the same place than the original
   orthogonal reflection section (that is, they do not rely on results proven
   later in \cite{adamek_rosicky}, in which case the argument could be circular).

   Interestingly enough, the proof in \cite{adamek_rosicky} entails the following:
   \begin{conjecture}
    Any cocontinous functor $\Cont\lambda(A)\rightarrow B$ has a right adjoint,
    which is roughly a nerve functor.
    Thus, anticipating Proposition~\ref{prop:pres-repr},
    \textbf{any cocontinuous
    functor out of a presentable category has a right adjoint}.
   \end{conjecture}
   I write it as a conjecture because \cite{adamek_rosicky} assumes $B$ is
   cocomplete, but I don't think it is important.
   \begin{proposition}
     \label{prop:pres-repr}
     (\cite[Representation Theorem, 1.46]{adamek_rosicky})
     A category $C$ is $\lambda$-presentable if and only if there exists a small
     category $A$ such that $C$ is equivalent to $\Cont\lambda(A)$.
     \end{proposition}
     \begin{conjecture}
       In Proposition~\ref{prop:pres-repr} above, the categories $A$ that work are exactly the
       full subcategories generated by a strong generator of $\lambda$-presentable objects.
     \end{conjecture}
       I should check that $C$ is equivalent to $\Cont\lambda(A)$ for a
        strong generator $A$ of $C$ consisting of $\lambda$-presenting objects.
        Anyway, we have the following lemma:
     \begin{lemma}
       \label{l:cont-repr-strong-gen}
      Representable functors in $\Cont\lambda(A)$ form a strong generator
      consisting of $\lambda$-presentable objects.
     \end{lemma}
     The only thing missing for $\lambda$-presentability of $\Cont\lambda(A)$ is
     that it is cocomplete, and this is proven in \S\ref{ss:ortho}.

       Regarding the proof, it is straightforward to check that representable presheaves form a strong
       generator.
       That they are $\lambda$-presentable relies on the following lemma:
       \begin{lemma}
         \label{l:cont-lambda-directed-colim}
         $\Cont\lambda(A)$ is closed under $\lambda$-directed colimits, which
         are computed pointwise.
         \end{lemma}
         In fact, $\Cont\lambda(A)$ is even cocomplete as we will see in
         \S\ref{ss:ortho}. Colimits in
         $\Cont\lambda(A)$ are computed pointwise exactly when the pointwise colimit
         is in $\Cont\lambda(A)$ (this is a general argument working for any full subcategory).
         Then, this lemma results from
           the fact that $\lambda$-directed colimits commute with
           $\lambda$-small limits in $\Set$.
     % It is very easy to prove that representable presheaves form a strong
     % generator.
     % Now to prove that
     % The proof is defered to \S\label{ss:ortho}
           \begin{proposition}
             (\cite[Representation Theorem 1.45]{adamek_rosicky})
             \label{proof:reflective-subcat-pres}
             A category $C$ is $\lambda$-presentable
             if and only if it is a reflective subcategory of a presheaf
             category, closed under (pointwise) $\lambda$-directed colimits.
           \end{proposition}
           This follows from Proposition~\ref{prop:pres-repr} and the following result:
           \begin{proposition}
              (\cite[Theorem 1.39]{adamek_rosicky})
              Let $C$ be a reflective subcategory of a $\lambda$-presentable category $K$ such
              that the embedding preserves $\lambda$-directed colimits.
              Then $C$ is $\lambda$-presentable.
           \end{proposition}
             \begin{proof}
               Here we provide a different proof than \cite{adamek_rosicky},
               not relying on orthogonal reflection.

               $C$ is cocomplete by Lemma~\ref{l:reflect-cocomplete}. Now we have
               to give a set of $\lambda$-presenting objects generating $C$
               under $\lambda$-directed colimits. We take
                the image of $\Pres\lambda(K)$ by the left adjoint.
           \end{proof}
   
\subsection{Orthogonal factorization systems}
\label{ss:ortho}
This section is somehow the counterpart of
orthogonal reflection section of \cite{adamek_rosicky}.
It relies on the observation that orthogonality classes there correspond
to fibrant objects for a cofibrantly generated orthogonal factorization system.
Thus, the transfinite induction that they do correspond to the usual small
object argument (although usually performed for weak factorization systems).

The goal here is to show that $\Cont\lambda(A)$ is cocomplete (and even
$\lambda$-presentable): this is the missing part of Lemma~\ref{l:cont-free}.
More precisely, we show that $\Cont\lambda$ is a reflective subcategory of $\hat{A}$.
We don't recall what orthogonal factorization systems are.

\begin{proposition}
 Given a small set $I$ of morphisms in a presentable category,
 Then $({}^\bot (I^\bot), I^\bot)$ is an orthogonal factorization system.
\end{proposition}
\begin{proof}
  The main point of the proof is the construction of the factorization.
  This is done by a transfinite construction.

This is claimed by the nlab page on orthogonal factorization systems, although I
haven't found a proper reference for it (maybe Kelly's transfinite paper, as
this result is also mentionned on the related nlab page).
\end{proof}
\begin{conjecture}
  \label{conj:fibrant-reflective}
 Given an orthogonal factorization system, the full subcategory of fibrant
 objects is reflective (the left adjoint simply factorizes the
 terminal morphism). 
\end{conjecture}
\begin{proof}
  I did not find a proper reference for this. I remember reading stuff about
  idempotent monads related to orthogonal factorization systems.
\end{proof}
\begin{corollary}
  \label{cor:cont-refl}
 $\Cont\lambda(A)$ is a reflective subcategory of $\hat{A}$ (and hence is
 cocomplete, by Lemma~\ref{l:reflect-cocomplete}).
\end{corollary}
This relies on the following observation:
\begin{remark}
  A presheaf $F:A^o \rightarrow \Set$  preserves a colimit $\int^i D(i)$ if and
  only if
  $F\rightarrow 1$ is right orthogonal to $y_{\int^i D(i)}\rightarrow \int_i y_{D(i)}  $.
\end{remark}
\begin{remark}
  Anticipating the sketch section, we can see why models of limit-sketches gather
  into a locally presentable category. For colimit sketches, we would consider
  left orthogonality rather than right orthogonality.
\end{remark}
\begin{corollary}
 $\Cont\lambda(A)$ is $\lambda$-presentable.
\end{corollary}
\begin{proof}
  It is cocomplete (Corollary~\ref{cor:cont-refl})
  and has a strong generator consisting of representables
  functors (Lemma~\ref{l:cont-repr-strong-gen}). Thus, it is $\lambda$-presentable according to
Definition~\ref{d:pres-strong-gen}.
\end{proof}


\subsection{Fun facts about orthogonal factorization systems}
In fact, orthogonal factorization systems are particular weak factorization
systems (at least if the category has pushouts)! They are such that the cofibrations are closed under a certain
pushout construction (and there is probably a dual statement for fibrations).

% \begin{lemma}
%  If a morphism is epimorphic, then there exists at most one filling for a square
%  where it is on the left.
% \end{lemma}
% \begin{lemma}
%   If a cocone consists of jointly epimorphic morphisms (in particular, if one of
%   the morphism is epimorphic), then the universal morphism from the colimit is epimorphic.
% \end{lemma}
% \begin{corollary}
% \end{corollary}
\begin{lemma}
  Let $f:A \rightarrow B$  be a morphism. Consider the universal morphism $B+_A B
  \rightarrow B$ (through the cocone consisting of identities), and a morphism
  $g : C \rightarrow D$.
  A commutative square
  \[
    \xymatrix{
      B+_A B \ar[r]\ar[d] & C \ar[d]^{g} \\
      B \ar[r]& D
    }
  \]
  % (which has at most one filling by the previous lemmas)
  is equivalently given by (where $g:C\rightarrow D$ is considered as given) two
  fillings of a commutative square
  \[
    \xymatrix{
      A \ar[r]\ar[d]_f & C \ar[d]^{g} \\
      B \ar[r]& D
    }
  \]
  % \begin{itemize}
  % % \item two morphisms $u_1,u_2:B \rightarrow C$ such that
  % %   \item $u_1 \circ f = u_2 \circ f$;
  % %   \item and  $g \circ u_1  = g\circ u_2 $ (which is the bottom arrow)
  %  % \item a morphism $v : B \rightarrow D$;
  %  %   \item such that $g \circ u_i  = v $
  % \end{itemize}
  A filling is then a witness that the two fillings are equal.
\end{lemma}
Thus, an factorization system which is closed under these codiagonals
is an orthogonal factorization system.
\section{Accessible categories}
We have already introduced accessible categories in Definition~\ref{d:accessible}.
\begin{question}
  How do you show the equivalence between different definitions of accessible
  categories? and the fact that any presentable category is accessible?
\end{question}
\begin{definition}
According to the nlab (page on sketch), an accessible category is
a full subcategory of a presheaf category that’s closed under $\kappa$-filtered colimits for some $\kappa$.
\end{definition}
\begin{remark}
 Contrary to what happens with presentability,
 a $\lambda$-accessible category may not be $\kappa$-presentable, for $\kappa > \lambda$.
 But there always exist such a $\kappa$ for which it is true.
 See the section ``Raising the index of accessibility'' in \cite{adamek_rosicky}.
 They introduce in \cite[2.12]{adamek_rosicky} a transitive relation ``sharply smaller''
 between two cardinals, for which it is true:
 \begin{itemize}
 \item 
   a cardinal is always
   sharply smaller than its successor;
   \item $\omega$ is sharply smaller than any other regular cardinal;
     \item if $\lambda\leq \mu$, then $\lambda$ is sharply smaller than
       the successor of $2^\mu$;
      \item as a consequence (not sure why) of the last item, for any set of cardinals, there
        exists a cardinal which is sharply greater than any cardian in this set.
 \end{itemize}
\end{remark}
As a consequence of this remark:
\begin{proposition}
  For any family of accessible categories,
  there exists arbitrary large regular
 cardinals $\mu$ such that they are $\mu$-accessible.
\end{proposition}
It often happens that small categories are accessible:
% (recall that the free split idempotency completion is called cauchy completion and
% gives a criterion to compare presheaf categories)
\begin{lemma}
 If a category has equalizers or coequalizers, then it has split idempotents.
\end{lemma}
\begin{proof}
 The splitting, according to nlab, consists in taking the (co)equalizer of the
 idempotent with the identity morphism.
\end{proof}
\begin{lemma}
  \cite[2.4]{adamek_rosicky}
  Each accessible category has split idempotents. 
\end{lemma}
\begin{proof}
 The diagram consisting of one object, the identity and the idempotent is
 $\lambda$-filtered. The colimit provides a splitting.
\end{proof}
\begin{proposition}
  \cite[2.6]{adamek_rosicky}
  Every small category with split idempotents is accessible. 
\end{proposition}
\begin{question}
  When $C$ is $\lambda$-presentable,
 $\Pres\lambda(C)$ has small $\lambda$-colimits and thus is accessible.
 What is $\Pres{}(\Pres\lambda(C))$ then?
\end{question}
\begin{remark}
  The Cauchy completion is an idempotent monad which freely adds split idempotents.
  I don't know if it will be relevant at some point, but it is known that
 two presheaves categories are equivalent if and only if the base categories have equivalent
 Cauchy completions.
\end{remark}


\begin{proposition}
  (\cite[Remarks 2.2, (3)]{adamek_rosicky})
  Any object $c$ of a $\lambda$-accessible category is presentable
  (although not necessarily $\lambda$-presentable).
\end{proposition}
A bound on the involved cardinal is the cardinal of the set of morphisms from any
element set of the generating set of presentable objects to $c$.
The proposition indeed follows from the following lemma, and the fact that $\lambda$-presentable
objects are also $\kappa$-presentable, for $\kappa>\mu$
\begin{lemma}
  (\cite[1.16]{adamek_rosicky})
 A $\lambda$-small colimit of $\lambda$-presentable objects is $\lambda$-presentable.
\end{lemma}
There is a not obvious converse to this fact:
\begin{proposition}
  (\cite[Proposition 2.3.11]{accessible})
  $\mu$-presentable objects are $\mu$-small $\lambda$-directed colimits of
  $\lambda$-presentable objects.
\end{proposition}
\begin{remark}
 The fact that a full and faithful embedding between two accessible categories preserves
 $\lambda$-directed colimits for some $\lambda$ is undecidable \cite[2.17.(3)]{adamek_rosicky}.
\end{remark}

\begin{proposition}
  (\cite[2.24]{adamek_rosicky})
  \label{prop:flat-functor}
 Let $F:A \rightarrow \Set$ be a functor. Then the opposite of its category of elements $\int F$
 is $\lambda$-filtered (that is, every $\lambda$-small diagram has a cone in
 $\int F$) if and only if it left Kan
 extension $\hat{A}\rightarrow \Set$ along yoneda preserve $\lambda$-small limits.

 In this case, as $F$ is a colimit over $(\int F)^o$, 
 $F$ is a $\lambda$-filtered colimit of representable presheaves.
\end{proposition}
\begin{remark}
  Such a functor is called \emph{flat}.
  We can restrict to $\lambda$-small limits of representable presheaves in the
  statement above.
  We also know from Lemma~\ref{l:free-cocomp} that this left Kan extension is
  cocontinuous.
\end{remark}
% The direct implication of this proposition is easy as we will argue. The converse relies on the
% following lemma:
% \begin{lemma}
%   \label{l:lkan-preserve}
%   Let $F: A \rightarrow \Set$ be a functor such that
%   its left Kan extension along yonedas preserve limits over diagrams on a
%   category $D$.
%   Then, for any diagram $K : D\rightarrow A$,
%   the canonical morphism
%   $\int^a F(a)\times Cone(a,K) \rightarrow \lim (FK)$ is an isomorphism, 
% where $Cone(a,K)$ is the set of
%   cones for $K$ with tip $a$. As an easy consequence,
%   $F$ also preserves $\lambda$-small limits.
% \end{lemma}
% \begin{remark}
%   With Proposition~\ref{prop:flat-functor} in mind, this means that any
%   $\lambda$-filtered colimit of representables presheaves preserve
%   $\lambda$-small limits.
%   The converse is not true. Otherwise, 
%   it would imply that a presheaf preserves $\lambda$-small limits if and only if
%   it a $\lambda$-directed colimit of representable presheaves, and thus,
%   anticipating some furhter theorem TODO, accessible categories would be the
%   same as presentable categories.
% \end{remark}
% \begin{proof}[Proof of Lemma~\ref{l:lkan-preserve}]
  
  Recall that $\Lan_yF(K) = \int^a F(a)\times K(a)$.
 We are sloppy and use equality where isomorphism would be more appropriate.

  Let $K : D \rightarrow A$.
  Then, $\lim F K = \int_d F(K(d)) = \int_d\int^a F(a)\times  A(a, K(d)) =
  \int_d \Lan_yF( A(-, K(d)))$ which by hypothesis is
  $ \Lan_yF(\int_dA(-,K(d)))=\int^a F(a)\times \int_dA(a, K(d))$.
  To conclude,
  $\int_dA(a,K(d))$ is the set of natural transformations between
  $a$ (as a constant functor) and $K$: that is, it is the set of cones.
% \end{proof}
% \begin{remark}
%   \label{r:lkan-preserve-cocone}
%   Let $F: A \rightarrow \Set$ be a functor such that
%   its left Kan extension along yonedas preserve limits over diagrams on a
%   category $D$. 
%   Then, for any diagram $K : D\rightarrow A$,
%   Lemma~\ref{l:lkan-preserve-cocone} implies that given any
%   element $x\in\lim FK$, there exists a $K$-cone in $A$ with tip $a$ and an
%   element $y\in F(a)$ such that $x$ is the image of $y$ by the universal arrow
%   into $\lim FK$.
% \end{remark}
\begin{proof}[Proof of Proposition~\ref{prop:flat-functor}]
It is clear that if $(\int F)^o$ is $\lambda$-filtered then the left Kan extension preserves
$\lambda$-small limits, because $F$ is a colimit over $(\int F)^o$, and
$\lambda$-small limits commute with $\lambda$-directed colimits in $\Set$.
Let us prove the converse and suppose that the left Kan extension preserve
$\lambda$-small limits of representable presheaves.
We show that any $\lambda$-small diagram in $\int F$ has a cone.
  
 By universal property of $\int F$ as the comma category $1/F$, for any category
 $D$, a functor $D \rightarrow \int F$ is equivalently given by:
 \begin{itemize}
 \item a functor $K : D  \rightarrow A$,
   \item and an object $\alpha\in \hom(1,FK))$.
 \end{itemize}
 Note that $\hom(1,FK) = \lim (FK)$. We are going to show that if the left
 Kan extension of $F$ preserves $D$-shaped limits, then $\lim(FK)$
 is canonically isomorphic to
 $\int^a F(a)\times Cone (a, K)$, where $Cone(a, K)$ is the set of cones for $K$
 with tip $a$.
Then, any antecedent of $\alpha$ through the surjection 
$\amalg_a F(a)\times Cone (a, K) \rightarrow \int^a F(a)\times Cone (a, K)$
provides a suitable cone.

Recall that $\Lan_yF(K) = \int^a F(a)\times K(a)$.
We are sloppy and use equality where isomorphism would be more appropriate.
Now, $\lim F K = \int_d F(K(d)) = \int_d\int^a F(a)\times  A(a, K(d)) =
\int_d \Lan_yF( A(-, K(d)))$ which by hypothesis is
$$ \Lan_yF(\int_dA(-,K(d)))=\int^a F(a)\times \int_dA(a, K(d)).$$
To conclude,
$\int_dA(a,K(d))$ is the set of natural transformations between
$a$ (as a constant functor) and $K$: that is, it is the set of cones.
% This provides the required cone.

% $\hom(D \rightarrow 1 \xrightarrow{F}{} \hat{A}, A \rightarrow{\hat{A}})$,
% where this hom category is computed in the $2$-slice category over $\hat{A}$.
% More precisely, 
 % defined as follows:
 % \begin{itemize}
 % \item objects are functors $K : D  \rightarrow A$ equipped with a
 %   an object $\alpha\in \lim FK (= \hom(1,FK))$
 %   (that is, by Lemma~\ref{l:lkan-preserve}, an element of $\int^a F(a)\times Cone (a, K)$);
 %   % a natural transformations $\alpha : 1 \Rightarrow FK$ (that is, an element in
 %   % the $\Set$-limit of $FK$);
 %   \item a morphism between $(K, \alpha)$ and $(K',\alpha')$ is a natural
 %     transformation $\beta : K \Rightarrow K'$ such that
 %     $\alpha' = \lim (F\beta) (\alpha)$.
 %     % \circ
 %     % \beta F$.
 % \end{itemize}

 % Let us fix the functor $K$. To give a diagram in $\int F$, the missing part is
 % al element $\alpha\in \lim FK$.

 % Next, a cone on a category $D$ in a category $C$ is given by a diagram
 % \begin{tikzcd}
 %   D \ar[r, bend  left=60, "c"{name=U}]
 %   \ar[r, bend right=60, "J"{name=D,below}]
 %   & C \ar[Rightarrow, from=U, to=D, % "\varphi" description
 %   ]
 % \end{tikzcd}
 % where $J$ is the diagram and $c$ is a constant functor induced by an object $c$.

 % In the case above where $C = \int F$, a cone is given by:
 % \begin{itemize}
 % \item an object $a'$  of $A$ with an element $x\in F(a')$
 % \item a functor $K : D \rightarrow A$;
 %   \item an element $\alpha \in \int^a F(a)\times Cone(a, K)$;
 %       \item a natural transformation $\beta : K \rightarrow a'$, that is, an
 %         element of $Cocone(K, a')$;
 %       \item such that the image of $\alpha$ by $\int^a F(a)\times Cone(a, K) \xrightarrow{\beta}{}
 %         \int^a F(a)\times A(a, a')^D = F(a')^D$ is the diagonal uplet
 %         consisting of $x$.
 % \end{itemize}
\end{proof}
\begin{remark}
 The 
 proof of Proposition~\ref{prop:flat-functor}
 does not constructively provide a cone, as it requires to choose an
 antecedent of some element by a surjective function.
\end{remark}

There is an alternative definition of accessible categories as free cocompletions.
In the following, $A$ denotes a small category.
\begin{definition}
  Let $U_{\Dir\lambda}$ denotes the forgetful functor from 
  categories with $\lambda$-directed colimits and functors preserving them to
  the category of small categories.
  We denote $\Dir\lambda(A)$ the full subcategory of $\hat{A}$ consisting of
  $\lambda$-directed colimits of representables presheaves.
\end{definition}
\subsection{Things to know}
\begin{theorem}
	\cite[2.47]{adamek_rosicky}
	An accessible category is complete if and only if it is cocomplete (in this case, it is locally presentable).
\end{theorem}
\section{Gabriel-Ulmer}
What is the category of algebras for the monad $\Pres\lambda\circ\Cont\lambda$?

  \section{Presentable categories (old version)}
  Let $\lambda$ be a regular cardinal.
  \begin{definition}
  A $\lambda$-presentable category is a
  category $C$ which is a free $\lambda$-cocompletion
  $\Cont\lambda(\Pres\lambda(C))$ of
  some small full subcategory $\Pres\lambda(C)$
     \cite[Representation theorem 1.46]{adamek_rosicky}.
	  Then, $\Pres\lambda(C)$ consist of \emph{$\lambda$-presentableobjects}
  \end{definition}

  Here, in this free $\lambda$-cocompletion, the right adjoint would start
  from 
  cocomplete (not $\lambda$-cocomplete!) categories and go to the category of
  categories and $\lambda$-cocontinous functors (i.e., functors preserving any
  $\lambda$-small colimit, not only the directed ones).
  \begin{remark}
	  $\Cont\lambda$ is 
	  obtained by restricting the category of presheaves to those preserving $\lambda$-small limits (as the yonedas do). Beware that colimits do not compute as in the total presheaf category (I can't find an argument for its cocompleteness in \cite{adamek_rosicky} by the way).
	  Is any functor still a canonical colimit of yonedas in $\Cont\lambda$?
	  The induced cocomplete functor is obtained by restricting the universal functor from $\hat{A}$ (because of the universal property of $\hat{A}$. Note: we get a cocomplete functor  from 
	  $\hat{A}$ to $\Cont\lambda$ by universal property of $\hat{A}$ .
	  This is obviously the left adjoint to the inclusion functor
	  (indeed, any cocontinuous functor out of $\hat{A}$ has a left adjoint, using yoneda to get the value of it)
  \end{remark}
  \begin{remark}
	  The remark after
     \cite[Representation theorem 1.46]{adamek_rosicky} says that a $\lambda$-presentable
	  category $K$ is equivalent to $\Cont\lambda(\Pres\lambda(K))$.
  \end{remark}
  \begin{remark}
	  I suspect that $A$ consists of presentable objects in $\Cont\lambda(A)$, but it can be less that all of them.
  \end{remark}

  Compare with the characterization of $\lambda$-accessible category:
  \begin{definition}
	  A $\lambda$-accessible category is a cocompletion $\Dir\lambda(A)$ of some small full subcategory $A$ with respect to $\lambda$-directed colimits
  \cite[Representation theorem 2.26]{adamek_rosicky}. 
  % Equivalently, it is a category which is equivalent to a subcategory of
  % functors from some small category $\CC$ to $\Set$ preserving
  % colimits and $\lambda$-small limits.
  \end{definition}
  Here, we think of the forgetful functor from categories with $\lambda$-directed colimits and functors preserving them to the category of categories as the right adjoint functor.
  The free stuff is obtained by restricting the category of presheaves to $\lambda$-directed colimits of yonedas.
  \begin{question}
   Is it clear that a $\lambda$-presentable category is $\lambda$-accessible
   with these characterizations? Not really...
	  I think the key point is to identify the requirements on $A$ to have $\Dir\lambda(A) = 
	  \Cont\lambda(A)$.
  \end{question}
  \begin{remark}
	  It is clear that $\lambda$-directed colimits of yonedas preserve $\lambda$-small colimits
	  (because $\lambda$-small limits commute with $\lambda$-directed colimits in $\Set$),
	  so $\Dir\lambda(A)\subset \Cont\lambda(A)$.
	  So I guess, we can always extend a $\lambda$-accessible category with the other presheaves that preserves small limits, and then we get a $\lambda$-presentable category.
  \end{remark}
  \begin{remark}
	  The first remark following
  \cite[Representation theorem 2.26]{adamek_rosicky} says that given a $\lambda$-accessible category
	  $K$, we have $K = \Dir\lambda(\Pres\lambda(K))$.
  \end{remark}
  \begin{question}
	   As $\lambda$-accessibility differs from $\lambda$-presentability, it is not always the case that 
	   $(K =) \Cont\lambda(A)=\Dir\lambda(A)$. But as any presentable category is accessible, there exists $B$ such that $K = \Dir\lambda(B)$. And after the previous remarks, we know that we can choose $B = \Pres\lambda(K)$. And then, $\Cont\lambda(B) = \Dir\lambda(B)$.
	   In other words, $\Cont\lambda = 
	   \Dir\lambda \circ \Pres\lambda \circ \Cont\lambda$.
	   We also know that $\Cont\lambda\circ\Pres\lambda\Cont\lambda = \Cont\lambda$
	   and similarly for $\Dir\lambda$.

	   What is this $B$ with respect to $A$?
	   More precisely, what does $\Pres\lambda(\Cont\lambda(A))$ do ?
	   Maybe a $\lambda$-small completion??
	   In fact, what can we say about $\Pres\lambda(K)$, for $K$ $\lambda$-accessible or presentable? For presentability, thinking of the example of presheaves, where presentable objects are finite colimits of yonedas, we might guess that $\Pres\lambda(\Cont\lambda(A))$ or $\Pres\lambda(\Dir\lambda(A))$ is the category of $\lambda$-small colimits of elements of $A$ (one should be this one, and the other is then different otherwise every accessible category would be presentable).
	   We may guess that $\Pres\lambda\circ\Dir\lambda(A)$ is smaller that $\Pres\lambda\circ\Cont\lambda(A)$, because $\Dir\lambda(A) \subset \Cont\lambda(A)$, but I am not sure. Is $\Pres\lambda$ monotonous?
	   I think this relates to Gabriel-Ulmer duality!! In fact $\Pres\lambda(K)$ is always small-complete when $K$ is $\lambda$-presentable. But how does it fit with finite colimits of yonedas in the case of presheaves? Oh yes, it makes sense now.. Is there Gabriel-Ulmer duality for accessible categories?
  \end{question}
  Alternative and (not)

  \begin{definition}
    A \textbf{$\lambda$-presentable object} in a category $C$ is an object such
    that its coyoneda embedding preserves $\lambda$-directed colimits.
  \end{definition}
  It says no more that any morphism from such an object to a $\lambda$-directed
  colimit factors (essentially uniquely) through some object of the colimiting cocone.
  \begin{lemma}
   A $\lambda$-presentable object is also $\kappa$-presentable, for $\kappa > \lambda$.
  \end{lemma}
  \begin{proof}
	  Obvious, because a $\kappa$-filtered category is also $\lambda$-filtered.
  \end{proof}
  Note that for accessible categories, we have a similar statement
  \cite[2.26]{adamek_rosicky} where the adjunction happens between the category
  of $\lambda$-directed cocomplete categories and the usual category of categories.
  \begin{question}
   Can you show with these characterizations that a presentable category is
   accessible? 
  \end{question}
  \begin{definition}[Alternative definition]
  A $\lambda$-accessible category is defined as a category closed
  under $\lambda$-directed colimit and there is a small set of
  $\lambda$-presentable objects such that every object is a $\lambda$-directed
  colimit of objects of $A$.
  \end{definition}
  \begin{definition}
	  (Another def \cite[1.20]{adamek_rosicky})
    A locally $\lambda$-presentable category is a cocomplete category
    with a strong generator formed of $\lambda$-presentable objects.
  \end{definition}
  To go from the strong generator to the set of $\lambda$-presentable objects
  above: take $\lambda$-small colimits of objects in the strong generator.
  \begin{question}
    How do you show the previous characterization of accessible categories?
\end{question}
\begin{lemma}
   A locally $\lambda$-presentable category is also locally $\kappa$-presentable, for $\kappa > \lambda$.
\end{lemma}
\begin{proof}
  We consider the alternative definition of locally presentable category.
  The set of ''presenting'' $\kappa$-presentable objects is induced by noticing that
  $\lambda$-presentable objects are also $\kappa$-presentable, and form a strong
  generator. Then, we take $\kappa$-small colimits of them.
\end{proof}
\subsection{Cauchy completion}
(let us recall that the free split idempotency is called cauchy completion and
gives a criterion to compare presheaf categories)
\begin{theorem}
  \cite[2.4]{adamek_rosicky}
 Each accessible category has split idempotents. 
\end{theorem}
\begin{theorem}
 Every small category with split idempotents is accessible. 
\end{theorem}
\section{Finitely presentable categories}
Any presheaf category is finitely presentable \cite{adamek_rosicky}: the
representable functors form the strong generator, and thus the presentable
objects are the finite colimits of these representable functors.

\section{Gabriel-Ulmer}
This section is based on Exercise 1.s and Remark 1.46 of \cite{adamek_rosicky}.
\begin{definition}
  
\end{definition}

\begin{theorem}
  For each regular cardinal $\lambda$, we denote
  \begin{itemize}
  \item 
  $c_{\lambda}CAT$ the category of
small categories $\lambda$-complete and functor preserving these limits;
\item $lp_{\lambda}CAT$ the category of $\lambda$-presentable categories and
  continuous and $\lambda$-accessible functors.
  \end{itemize}
Then, the following functors are equivalences:
\begin{align*}
 R : c_{\lambda}CAT & \rightarrow lp_{\lambda}CAT
  \\
  C & \mapsto \Cont\lambda C^o
      \\
 L : lp_{\lambda}CAT& \rightarrow c_{\lambda}CAT 
  \\
  C & \mapsto \Pres\lambda(C)^o
\end{align*}
\end{theorem}
Actually, they even induce a biequivalence ($c_{\lambda}CAT$ and
$lp_{\lambda}CAT$ can be equipped with a 2-categorical structure).
\begin{question}
 Is it a strict 2-equivalence? I guess no, otherwise it would have been said. 
\end{question}

This theorem induces two (equivalent) contravariant functors
from the poset of regular cardinals (seen
as a category) to the category of small categories. Performing the Grothendieck construction yield
an equivalence between some category of presentable categories and some category
of small categories (fibered over the
category of regular cardinals).
\section{Sketches}
\begin{definition}
 A \textbf{sketch} is a small category $C$ with a family of (small) diagrams $F_i : D_i
 \to C$ together with a choice of a cone or a cocone for each of this diagram.
\end{definition}
\begin{definition}
  The \textbf{category of models of a sketch $(C,(F_i,c_i)_i)$} is the full
  subcategory of $[C,\Set]$ mapping any mapping any chosen cones and cocones to
  limits and colimits.
\end{definition}
\begin{theorem}
 Any accessible category is equivalent to the category of models of a sketch.
\end{theorem}
\begin{theorem}
 Any presentable category is equivalent to the category of models of a
 \textbf{limit sketch}, that is, a sketch with only cones (and no cocones).
\end{theorem}
\begin{question}
 How do you retrieve the regular cardinal from this description? Probably, it is
some cardinal bound over the size of the diagrams?
\end{question}
\begin{theorem}
  (nlab)
 A  category is $\lambda$-accessible iff it is equivalent to a full subcategory
 of a presheaf category closed under $\lambda$-filtered colimits.
\end{theorem}
\begin{question}
  How does the raising of the indexing regular cardinal translates into this definition?
\end{question}
\begin{theorem}
 A category is presentable if it is equivalent to a reflective subcategory
 of a presheaf category.
\end{theorem}
Note that a accessible category is presentable
precisely if the embedding into the presheaf category has a left adjoint, indeed:
\begin{theorem}\cite[Proposition 2.4.8]{accessible} or \cite[2.23]{adamek_rosicky}
  Any right adjoint functor between accessible categories is accessible.
\end{theorem}

% Hence, for presentable category, we get a left adjoint to the embedding into the
% category of presheaf
\begin{theorem}
  \cite[Corollary 2.62]{adamek_rosicky}
  For any sketch, there exists another one whose cones/cocones are all
  limiting/colimiting and has an equivalent category of models.
\end{theorem}
By Gabriel-Ulmer duality, we already know that for limit sketches: you can
take the $\lambda$-complete category corresponding to the presentable category
and take all the $\lambda$-small limits as cones.
\begin{question}
 What does exactly this construction on a sketch $S$ do? 
 % I have noted somewhere that it is given by the factorization of  $S\to
 % \Set^{S}\to Mod(S)$
 % where the last functor is the loù la dernière flèche est l'adjoint à gauche de l'inclusion de Mod dans Set^{S}. L'inclusion de S dans ce sketch réalisant est alors l'identité sur les objets.
\end{question}
\section{GATs and EATs}
generalized algebraic theories (or gats) are defined in \cite{CARTMELL}, and
also more formally (and recently) in \cite{combinatorial-structure}


\subsection{QIITs are (particular) GATs}
It is not exactly clear what are GATs introduced by \cite{CARTMELL}, in particular what is the status of
equations of sorts.
GATs are like QIITs but with a possible infinite context. More concretely, a GAT
consists of a possibly infinite set of axiomatic judgments, and each judgement must be
wellformed (with respect to a finite subset of axioms).
Infinitary QIITs allow to capture some of them, but not all.
\begin{question}
Can we find a
counter example?
\end{question}

Also, in QIITs, we can have equality constructors taking equalities in
arguments, that we cannot do directly with GATs. But fortunately, such constructors can be encoded
by introducing new types in the universe which are the limits of other ones
(such incorporating the equations).
\begin{question}
\cite{CARTMELL}  claims that GATs are equivalent to EATs, but how to deal with
equations of sorts? How are they converted in the EAT version ?
\end{question}
Cartmell also proves
that they are equivalent to contextual categories, and for this he needs these
sorts equations
(see (iii) at the
top of p107 of his PhD manuscript).

Taichi sees two difficulties when trying to replace sort
equations with isomorphisms:
\begin{itemize}
\item If we replace a sort equation $A = B$ with a sort isomorphism $e : A \simeq B$, we also have to replace occurences of coercion along $A = B$ with
  applications of $e$.
  \item A naive replacement can create a different GAT. For
  example, consider the GAT consisting of a constant sort A and equation $A = A$
  (although this equation is redundant, it is a valid GAT). A model of
  $(A, A = A)$ is just a set, but a model of $(A, e : A \simeq A)$ is a set equipped with an
  automorphism. They do not seem to be equivalent.
\end{itemize}
I expect that 1. is possible, although I confess I don't know how to
do it in general.
For the second point, I would suggest the
following: given any sort equation $A = A$ that you can prove in the
theory induced by a GAT with sort equations, you get (I hope) an
automorphism on the translation of $A$ (in the translated GAT which
turns sort equations into isomorphisms). Then,  for each such sort
equation, add in the translated GAT that this automorphism should be
the identity.
\subsection{QIITs are QITs (as GATs are EATs)?}

No! As noticed by Ambrus, EATs allow equations in the arguments (i.e., partial
constructors), whereas QITs do not.
\section{Equational systems?}
\appendix
\section{Some useful categorical results}
\begin{lemma}
  \label{l:reflect-cocomplete}
  A reflective subcategory of a cocomplete category is cocomplete. 
\end{lemma}
\begin{proof}
  Take the colimit in the cocomplete category, and apply the left adjoint.
\end{proof}
As already mentioned, a colimit in the full subcategory does not coincide with the
one computed in the total category unless the latter is essentially in the subcategory.
\begin{proposition}
 In $\Set$, $\lambda$-filtered colimits are exactly those commuting with any $\lambda$-small limit.
\end{proposition}
\begin{lemma}
\label{l:cocontinuous-lkan}
 Let $A$  be a small category, $C$ be full subcategory of $\hat{A}$
 containing representable presheaves, and $B$ be a cocomplete category.
 Then the left Kan extension of any functor $F : A\rightarrow B$ along the
 yoneda embedding into $C$ maps a presheaf $G$ to the coend $\int^a F(a)\times
 G(a)$.
 In particular, it is cocontinuous.
\end{lemma}
\begin{proof}
 Obvious when unfolding the coend formula of left Kan extension. 
\end{proof}
\begin{lemma}
  \label{l:unique-cocontinuous-lkan}
  Let $A$  be a small category, $C$ be a reflective subcategory of $\hat{A}$
  containing representable presheaves, and $B$ be a cocomplete category.
  Then the left Kan extension of any functor $F : A\rightarrow B$ along the
  yoneda embedding into $C$ is the unique cocontinuous functor which
  postcomposed with the yoneda embedding yields $F$.
\end{lemma}
\begin{proof}
  Suppose you have two such functors $F$ and $F'$ and let us denote $L$ the left
  adjoint to the embedding of $C$ in $\hat{A}$. 
  As $\hat{A}$ is the free cocompletion, $F\circ L = F'\circ L$. But then, by
  precomposing with the full and faithful embedding $R$ and exploiting the fact
  that $L\circ R$ is isomorphic to the identity, we get the desired result.
\end{proof}
\bibliographystyle{unsrt}
\bibliography{bibli}
\end{document}

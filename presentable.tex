\documentclass{article}
% \usepackage[margin=1.5cm,landscape]{geometry}
\usepackage{proof}
\usepackage{amssymb}
\usepackage{amsmath}
\usepackage{amsthm}
\usepackage[utf8]{inputenc}
\usepackage{pdflscape}
\usepackage{multirow,bigdelim}
\usepackage{hyperref}
\usepackage{multicol}
\title{A note on (locally) presentable categories}
\author{Ambroise LAFONT}

\newcommand{\CC}{\mathsf{C}}
\newcommand{\DD}{\mathsf{D}}
\newcommand{\Cont}[1]{\text{Cont}_{#1}}
\newcommand{\Pres}[1]{\text{Pres}_{#1}}
\newcommand{\CwF}{\text{CwF}}
\newcommand{\T}{\mathcal{T}}
\newcommand{\U}{\mathcal{U}}
\newcommand{\id}{\text{id}}
\newcommand{\Set}{\text{Set}}
\newcommand{\im}{\text{Im }}
\newcommand{\bool}{\text{bool}}
\newtheorem{theorem}{Theorem}[section]
\newtheorem{definition}[theorem]{Definition}
\newtheorem{conjecture}[theorem]{Conjecture}
\newtheorem{question}[theorem]{Question}
\newtheorem{remark}[theorem]{Remark}
\newtheorem{lemma}{Lemma}
\begin{document}
\maketitle
% This is a report of my visit to Ambrus Kaposi in Budapest from 26th August to
% 30th August 2019, and various thoughts

\tableofcontents

A \emph{locally presentable category} (or just \emph{presentable} category) is a
category which is not too big and not too small: it is cocomplete, but is the
generated by a small set of objects.
The main reference for this beast is \cite{adamek_rosicky}.


A morphism between presentable categories is defined as cocontinuous functor in
Anel's and Joyal's Topo-logie.

Presentable categories can be defined as
cocomplete \emph{accessible categories}, or as complete accessible categories
(after nlab, the page on accessible categories and locally presentable categories).

An accessible category is defined as a $\lambda$-accessible category (you don't
need to know what it is at this stage) for some
regular cardinal $\lambda$. This a notion of $\lambda$-presentable category, and
a presentable categories is a $\lambda$-presentable category, for some regular
cardinal $\lambda$. Any $\lambda$-presentable category is also $\mu$-presentable
for any regular cardinal $\mu>\lambda$. This is not the case for accessible
categories! We only know, for them, that given any regular cardinal $\lambda$,
there exists $\kappa>\lambda$ such that any $\lambda$-accessible category is
also $\kappa$-accessible \cite[Theorem 2.11]{adamek_rosicky} (such a condition
is noted $\lambda \triangleleft \kappa$ and $\lambda$ is said sharply smaller,
and it is transitive).
In particular, $\omega\triangleleft \lambda$ for any (infinite) regular cardinal
$\lambda$, and a regular cardinal is always sharply smaller than its successor.
For each set of regular cardinals, there is one which is sharply greater than
all of them. A similar thing can be said of accessible functors.


Presentable categories satisfy nice properties.

\begin{definition}
 A functor is said \textbf{to have a rank}  if it has a rank $\lambda$,
 for some regular cardinal $\lambda$, that is, if it
  preserves $\lambda$-directed colimits.
  A functor is said ($\lambda$-)\textbf{accessible} if it has a rank ($\lambda$) and the domain/codomain
  are ($\lambda$-)accessible.
\end{definition}
\begin{theorem}
  \cite[1.66]{adamek_rosicky}
  A functor between presentable categories is right adjoint iff it is continuous
  and has a rank.
  \end{theorem}
  A similar version is true of left adjointness according to nlab (wihtout the
  accessibility condition because it is automatically satisfied).
  In fact, by looking at the proof of \cite[Representation theorem 1.45]{adamek_rosicky}, it seems that any cocontinous functor out of a presentable category
  has a right adjoint (because any presentable category is of the shape $\Cont\lambda(A)$, and in the proof mentioned above, they say that the cocontinuous induced functor to a cocomplete category has a left adjoint obtained easily using yoneda. I suspect cocompleteness of the target is not necessary here).

  \begin{remark}
	  Anticipating the fact that any presentable category is of the shape $\Cont\lambda(A)$,
	  a right adjoint to a presentable categorie is always a (kind of) nerve functor. This is seen, by postcomposing with the right adjoint inclusion in $\hat{A}$, and using yoneda to get the definition of this right adjoint.
  \end{remark}

  \begin{theorem}[nlab]
   Let $T$  be an accessible monad on a presentable category $C$.
   Then $C^T$ is presentable.
  \end{theorem}
  % Indeed, the right adjoint of a monadic adjunction

\section{Regular cardinals}
\begin{definition}
 A cardinal is regular if it is strictly greater than any smaller union of smaller cardinals.
\end{definition}
Any successor cardinal is regular. The existence of regular uncountable limit
cardinal (called weakly inacessible cardinal) is not known to be consistent with
ZFC. It is consistent to assume that any uncountable limit cardinal is not regular.

\section{Filtered/directed colimits}
A $\lambda$-filtered category is a category $\CC$ in which any diagram
$\DD\to \CC$ with $\DD$ $\lambda$-small has a cocone. A $\lambda$-directed
category is a poset which induces a $\lambda$-filtered category.

For any $\lambda$-filtered category $\CC$, there is a $\lambda$-directed one
$\DD$ with a cofinal functor $\DD\to\CC$. The existence of a cofinal functor
means that colimits with respect to $\CC$ is the same as colimits with respect
to $\DD$.

That is why we can restrict ourselves to $\lambda$-directed colimits, rather
than the more general $\lambda$-filtered colimits. We can also restrict to
chains for $\omega$-filtered colimits, although there is no cofinality argument
(see~\cite[Example 1.8]{adamek_rosicky}).
% Also, we have this wonderful theorem:
\begin{theorem}
 \cite[Corollary 1.7]{adamek_rosicky}:
 A category has $\omega$-filtered (or $\omega$-directed) colimits iff it has
 colimits of chains. For such categories $K$, a functor of domain $K$ preserve
 $\omega$-filtered colimits iff it preserves colimits of chains.
\end{theorem}

  \section{Presentable categories}
  Let $\lambda$ be a regular cardinal.
  \begin{definition}
  A $\lambda$-presentable category is a
  category $C$ which is a free $\lambda$-cocompletion
  $\Cont\lambda(\Pres\lambda(C))$ of
  some small full subcategory $\Pres\lambda(C)$
     \cite[Representation theorem 1.46]{adamek_rosicky}..
	  Then, $\Pres\lambda(C)$ consist of \emph{$\lambda$-presentableobjects}
  \end{definition}

  Here, in this free $\lambda$-cocompletion, the right adjoint would start
  from 
  cocomplete (not $\lambda$-cocomplete!) categories and go to the category of
  categories and $\lambda$-cocontinous functors (i.e., functors preserving any
  $\lambda$-small colimit, not only the directed ones).
  \begin{remark}
	  $\Cont\lambda$ is 
	  obtained by restricting the category of presheaves to those preserving $\lambda$-small limits (as the yonedas do). Beware that colimits do not compute as in the total presheaf category (I can't find an argument for its cocompleteness in \cite{adamek_rosicky} by the way).
	  Is any functor still a canonical colimit of yonedas in $\Cont\lambda$?
	  The induced cocomplete functor is obtained by restricting the universal functor from $\hat{A}$ (because of the universal property of $\hat{A}$. Note: we get a cocomplete functor  from 
	  $\hat{A}$ to $\Cont\lambda$ by universal property of $\hat{A}$ .
	  This is obviously the left adjoint to the inclusion functor
	  (indeed, any cocontinuous functor out of $\hat{A}$ has a left adjoint, using yoneda to get the value of it)
  \end{remark}

  Compare with the characterization of $\lambda$-accessible category:
  \begin{definition}
  A $\lambda$-accessible category is a cocompletion of some small full subcategory with respect to $\lambda$-directed colimits
  \cite[Representation theorem 2.26]{adamek_rosicky}. 
  % Equivalently, it is a category which is equivalent to a subcategory of
  % functors from some small category $\CC$ to $\Set$ preserving
  % colimits and $\lambda$-small limits.
  \end{definition}
  Here, we think of an
  adjunction between categories and categories with $\lambda$-directed colimits
  and functors preserving them.
  The free stuff is obtained by restricting the category of presheaves to $\lambda$-directed colimits of yoneds.
  \begin{question}
   Is it clear that a $\lambda$-presentable category is $\lambda$-accessible
   with these characterizations? Not really...
  \end{question}
  Alternative and (not)
  \begin{definition}
    
  \end{definition}

  \begin{definition}
    A \textbf{$\lambda$-presentable object} in a category $C$ is an object such
    that its coyoneda embedding preserves $\lambda$-directed colimits.
  \end{definition}
  It says no more that any morphism from such an object to a $\lambda$-directed
  colimit factors (essentially uniquely) through some object of the colimiting cocone.
  \begin{lemma}
   A $\lambda$-presentable object is also $\kappa$-presentable, for $\kappa > \lambda$.
  \end{lemma}
  \begin{proof}
	  Obvious, because a $\kappa$-filtered category is also $\lambda$-filtered.
  \end{proof}
  Note that for accessible categories, we have a similar statement
  \cite[2.26]{adamek_rosicky} where the adjunction happens between the category
  of $\lambda$-directed cocomplete categories and the usual category of categories.
  \begin{question}
   Can you show with these characterizations that a presentable category is
   accesible? 
  \end{question}
  \begin{definition}[Alternative definition]
  A $\lambda$-accessible category is defined as a category closed
  under $\lambda$-directed colimit and there is a small set of
  $\lambda$-presentable objects such that every object is a $\lambda$-directed
  colimit of objects of $A$.
  \end{definition}
  \begin{definition}
	  (Another def \cite[1.20]{adamek_rosicky})
    A locally $\lambda$-presentable category is a cocomplete category
    with a strong generator formed of $\lambda$-presentable objects.
  \end{definition}
  To go from the strong generator to the set of $\lambda$-presentable objects
  above: take $\lambda$-small colimits of objects in the strong generator.
  \begin{question}
    How do you show the previous characterization of accessible categories?
\end{question}
\begin{lemma}
   A locally $\lambda$-presentable category is also locally $\kappa$-presentable, for $\kappa > \lambda$.
\end{lemma}
\begin{proof}
  We consider the alternative definition of locally presentable category.
  The set of ''presenting'' $\kappa$-presentable objects is induced by noticing that
  $\lambda$-presentable objects are also $\kappa$-presentable, and form a strong
  generator. Then, we take $\kappa$-small colimits of them.
\end{proof}
\subsection{Cauchy completion}
(let us recall that the free split idempotency is called cauchy completion and
gives a criterion to compare presheaf categories)
\begin{theorem}
  \cite[2.4]{adamek_rosicky}
 Each accessible category has split idempotents. 
\end{theorem}
\begin{theorem}
 Every small category with split idempotents is accessible. 
\end{theorem}
\section{Finitely presentable categories}
Any presheaf category is finitely presentable \cite{adamek_rosicky}: the
representable functors form the strong generator, and thus the presentable
objects are the finite colimits of these representable functors.

\section{Gabriel-Ulmer}
This section is based on Exercise 1.s and Remark 1.46 of \cite{adamek_rosicky}.
\begin{definition}
  
\end{definition}

\begin{theorem}
  For each regular cardinal $\lambda$, we denote
  \begin{itemize}
  \item 
  $c_{\lambda}CAT$ the category of
small categories $\lambda$-complete and functor preserving these limits;
\item $lp_{\lambda}CAT$ the category of $\lambda$-presentable categories and
  continuous and $\lambda$-accessible functors.
  \end{itemize}
Then, the following functors are equivalences:
\begin{align*}
 R : c_{\lambda}CAT & \rightarrow lp_{\lambda}CAT
  \\
  C & \mapsto \Cont\lambda C^o
      \\
 L : lp_{\lambda}CAT& \rightarrow c_{\lambda}CAT 
  \\
  C & \mapsto \Pres\lambda(C)^o
\end{align*}
\end{theorem}
Actually, they even induce a biequivalence ($c_{\lambda}CAT$ and
$lp_{\lambda}CAT$ can be equipped with a 2-categorical structure).
\begin{question}
 Is it a strict 2-equivalence? I guess no, otherwise it would have been said. 
\end{question}

This theorem induces two (equivalent) contravariant functors
from the poset of regular cardinals (seen
as a category) to the category of small categories. Performing the Grothendieck construction yield
an equivalence between some category of presentable categories and some category
of small categories (fibered over the
category of regular cardinals).
\section{Sketches}
\begin{definition}
 A \textbf{sketch} is a small category $C$ with a family of (small) diagrams $F_i : D_i
 \to C$ together with a choice of a cone or a cocone for each of this diagram.
\end{definition}
\begin{definition}
  The \textbf{category of models of a sketch $(C,(F_i,c_i)_i)$} is the full
  subcategory of $[C,\Set]$ mapping any mapping any chosen cones and cocones to
  limits and colimits.
\end{definition}
\begin{theorem}
 Any accessible category is equivalent to the category of models of a sketch.
\end{theorem}
\begin{theorem}
 Any presentable category is equivalent to the category of models of a
 \textbf{limit sketch}, that is, a sketch with only cones (and no cocones).
\end{theorem}
\begin{question}
 How do you retrieve the regular cardinal from this description? Probably, it is
some cardinal bound over the size of the diagrams?
\end{question}
\begin{theorem}
  (nlab)
 A  category is $\lambda$-accessible iff it is equivalent to a full subcategory
 of a presheaf category closed under $\lambda$-filtered colimits.
\end{theorem}
\begin{question}
  How does the raising of the indexing regular cardinal translates into this definition?
\end{question}
\begin{theorem}
 A category is presentable if it is equivalent to a reflective subcategory
 of a presheaf category.
\end{theorem}
Note that a accessible category is presentable
precisely if the embedding into the presheaf category has a left adjoint, indeed:
\begin{theorem}\cite[Proposition 2.4.8]{accessible} or \cite[2.23]{adamek_rosicky}
  Any right adjoint functor between accessible categories is accessible.
\end{theorem}

% Hence, for presentable category, we get a left adjoint to the embedding into the
% category of presheaf
\begin{theorem}
  \cite[Corollary 2.62]{adamek_rosicky}
  For any sketch, there exists another one whose cones/cocones are all
  limiting/colimiting and has an equivalent category of models.
\end{theorem}
By Gabriel-Ulmer duality, we already know that for limit sketches: you can
take the $\lambda$-complete category corresponding to the presentable category
and take all the $\lambda$-small limits as cones.
\begin{question}
 What does exactly this construction on a sketch $S$ do? 
 % I have noted somewhere that it is given by the factorization of  $S\to
 % \Set^{S}\to Mod(S)$
 % where the last functor is the loù la dernière flèche est l'adjoint à gauche de l'inclusion de Mod dans Set^{S}. L'inclusion de S dans ce sketch réalisant est alors l'identité sur les objets.
\end{question}
\section{GATs and EATs}
generalized algebraic theories (or gats) are defined in \cite{CARTMELL}, and
also more formally (and recently) in \cite{combinatorial-structure}


\subsection{QIITs are (particular) GATs}
It is not exactly clear what are GATs introduced by \cite{CARTMELL}, in particular what is the status of
equations of sorts.
GATs are like QIITs but with a possible infinite context. More concretely, a GAT
consists of a possibly infinite set of axiomatic judgments, and each judgement must be
wellformed (with respect to a finite subset of axioms).
Infinitary QIITs allow to capture some of them, but not all.
\begin{question}
Can we find a
counter example?
\end{question}

Also, in QIITs, we can have equality constructors taking equalities in
arguments, that we cannot do directly with GATs. But fortunately, such constructors can be encoded
by introducing new types in the universe which are the limits of other ones
(such incorporating the equations).
\begin{question}
\cite{CARTMELL}  claims that GATs are equivalent to EATs, but how to deal with
equations of sorts? How are they converted in the EAT version ?
\end{question}
Cartmell also proves
that they are equivalent to contextual categories, and for this he needs these
sorts equations
(see (iii) at the
top of p107 of his PhD manuscript).

Taichi sees two difficulties when trying to replace sort
equations with isomorphisms:
\begin{itemize}
\item If we replace a sort equation $A = B$ with a sort isomorphism $e : A \simeq B$, we also have to replace occurences of coercion along $A = B$ with
  applications of $e$.
  \item A naive replacement can create a different GAT. For
  example, consider the GAT consisting of a constant sort A and equation $A = A$
  (although this equation is redundant, it is a valid GAT). A model of
  $(A, A = A)$ is just a set, but a model of $(A, e : A \simeq A)$ is a set equipped with an
  automorphism. They do not seem to be equivalent.
\end{itemize}
I expect that 1. is possible, although I confess I don't know how to
do it in general.
For the second point, I would suggest the
following: given any sort equation $A = A$ that you can prove in the
theory induced by a GAT with sort equations, you get (I hope) an
automorphism on the translation of $A$ (in the translated GAT which
turns sort equations into isomorphisms). Then,  for each such sort
equation, add in the translated GAT that this automorphism should be
the identity.
\subsection{QIITs are QITs (as GATs are EATs)}
Example:
$Con:\U, Ty:Con\to\U$ is translated as $ Con:\U,\int Ty : \U, con : \int Ty\to
Con$ (this is the idea of translating GATs into EATS, as in \cite{CARTMELL}).
\begin{question}
  Maybe this generalizes to HIITs and HITs?
\end{question}
\section{Equational systems?}
\bibliographystyle{unsrt}
\bibliography{bibli}
\end{document}
